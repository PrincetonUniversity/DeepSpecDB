% Contributions:
% verified C implementation of BST -- the first?
% technique for connecting locks with abstract state/atomicity
% VST-style decomposition of proofs into C-to-local-abstract and local-to-global-abstract
% lock-free implementation?


The binary search tree (BST) is a common implementation of an ordered
map, a widely used data structure. The concurrent version of the
ordered map forms the bedrock of many parallel programs. We formally
verified the functional correctness of two versions of fine-grained
concurrent binary search trees (CBST) written in C. One adopts the
hand-over-hand locking technique \cite{bayer1977}, the other is lock
free by using atomic primitives such as compare and swap (CAS). Both
CBST implementations support insert, lookup, and delete operations; %or do we have lock-free without delete?
they also share the ``same'' specifications to some extent in our
verification. All the proof is machine-checked in Coq. %Say something about existing proofs of fine-grained data structures and how we compare.

Our specifications use the logical atomicity introduced in the TaDA
logic \cite{tada}, in the form of \emph{atomic Hoare
triples}. Intuitively, an atomic Hoare triple $ \langle
P \rangle\,\mathbb{C}\, \langle P \rangle$ means the program
$\mathbb{C}$ ``atomically updates'' from $P$ to $Q$. The program may
actually take multiple steps, but evey step before the atomic update
(linearization point) must preserve the assertion $P$. Meanwhile, the
concurrent environment may also update the state before the
linearization point, as long as the states satisfies $P$. The
assertion $Q$ must become true at the linearization point, then the
environment can do whatever it likes. There is no guarantee that $Q$
is still preserved when $\mathbb{C}$ returns. For example, the
specification of our \texttt{insert} operation may be explained as
follows: during the execution of \texttt{insert}, there is
always \emph{some} BST, and at some point the \texttt{insert} will
take a BST $t$, insert a value with a certain key, and then eventually
returns (meanwhile other threads may have further modified the
inserted tree).

We employ the Verified Software Toolchain (VST) \cite{plfcc} to verify
the correctness of CBST. Although the concurrent separation logic
(CSL) has been formalized in VST rooted on the work of Hobor et
al.\ \cite{oraclesematic}, we extend it with two descendants of CSL so
as to accomplish the verification. One is the logical atomicity
mentioned above; the other is the higher-order ghost state in the
style of Iris \cite{higherorderghoststate}. The ghost states are used
to contruct both the global invariants and the local state in our
proofs. They will be further discussed in \S\ref{correctness}.

We highlights a few innovative aspects about our verification of CBST:
\begin{description}
\item [Range ghost] We abstract the BST via a pair of values
called \emph{range} which represents the lower-bound and upper-bound
of keys on each node to reason about the BST in the current
settings. We prove that range with the merging operation forms a
partial commutative monoid (PCM) so that it can be encoded in ghost
states.

\item [\textsf{sync\_inv} pattern] It is a particular approach
combining locks with general invariants to solve the dilemma caused by
the fine-grained locking mechanism: we do not have a lock to control
the access of the entire BST nor can we access the state of the BST
via atomic operations.
\end{description}

%% Do we have contributions, and are they sufficient? Range ghost is adapted from flow interfaces. Sync_inv pattern isn't totally ours, though it might be instructive w.r.t. combining traditional and Iris-style CSLs, and it might be worth explaining in more detail. We're working on "actual" code, but missing a few pieces to make our proofs foundational. (but maybe it's worth highlighting the extra work we do to work on actual code) We could promote this as the first working example of Iris+VST. We don't yet have any demonstrations on larger or interesting data structures, and this alone was quite a bit of work. The delete logic is rather interesting. Nothing to say about flow interfaces yet, but maybe soon.


Our specific contributions are:
\begin{itemize}
\item To the best of our knowledge, this is the first mechanized
verification of an concurrent search-based data structure written in a
real programming language. %should probably clarify what we mean by "real"

\item We illustrate how to incorporate the CSL in VST, the higher-order
ghost state in the style of Iris, and the logical atomicity from the
TaDA logic together to verify the CBST.
\end{itemize}

We introduce the background about the verification of concurrent C
programs in \S\ref{background}, including the tool-chain VST and Iris,
the concept of ghost states, global invariants, and atomic
specifications. The thread-safety proofs of operations on CBST are
first explained in \S\ref{safety}, where we show the \emph{recursive}
lock pattern for hand-over-hand locking mechanism. The functional
correctness proofs are presented in \S\ref{correctness}. We detail the
use of the \textsf{sync\_inv} pattern, the combination of recursive
lock invariants and ghost states together in the atomic
specifications, and the proof skeleton of each operation on CBST. The
related work is discussed in \S\ref{related}. We end with the
conclusion of our work in \S\ref{conclusion}.

