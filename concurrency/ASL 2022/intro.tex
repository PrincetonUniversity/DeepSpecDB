% Contributions:
% verified C implementation of BST with atomic specs -- the first? VeriFast did it, but not atomic specs
% technique for connecting locks with abstract state/atomicity
	% note that the flow interfaces team sort of does atomicity for locks, but they assume an atomic spec for locks that's also application-specific? no, they implement them in the TaDA style and prove an atomic spec.
% VST-style decomposition of proofs into C-to-local-abstract and local-to-global-abstract -- this corresponds to flow interfaces/CSS templates, basically
% delete proofs -- interesting interplay between non-visible range increase and lock-protected range modification
% lock-free implementation?

% start with atomic specs for cglock implementation, where the local ghost state *is* the global ghost state!

Concurrent separation logic (CSL)~\cite{csl} is a useful tool for proving correctness of concurrent programs. While early iterations focused on data structures implemented with locks~\cite{gotsman,oraclesematic} and specifications involving memory safety and data-race-freedom, more recent logics are able to prove linearizability-style functional correctness~\cite{tada} of lock-free implementations that use low-level atomic memory operations~\cite{iris}, even under weak memory models~\cite{gps}%update cite
. In particular, \emph{logical atomicity}~\cite{tada} has become the gold standard for concurrent data structure specifications, capturing the idea that data structure operations should appear to take place instantaneously with no externally visible intermediate states. The effects of a logically atomic operation become visible at a \emph{linearization point}, a single instruction (usually an atomic memory operation or call to a logically atomic function) that publishes the new state of the data structure.

Intuitively, a critical section in a lock-based implementation serves the same role as an atomic instruction in a lock-free data structure, and it is not hard to identify linearization points in lock-based implementations---the effects of an update become visible when the lock protecting the updated data is released. The translation of this intuition into proof, however, depends on the specifications used for lock operations. The CSL literature contains two distinct specifications for lock acquire and release:
\begin{enumerate}
\item In Gotsman/Hobor-style specs, each lock is associated with a lock invariant $R$, which is gained by \texttt{acquire} operations and restored by \texttt{release} operations.
\item In TaDA-style specs, \texttt{acquire} atomically moves a lock from the unlocked state to the locked state, and \texttt{release} does the reverse.
\end{enumerate}
Style 1 can be derived from style 2, and is also a natural fit for atomicity proofs of larger data structures, so it has been the style of choice for most recent verifications~\cite{tada-live,templates}. However, unpublished work by Zhang~\cite{atomic-syncer} shows that style 1 can also be used to prove atomic specifications for data structures.

In this paper, we adapt this approach to the Verified Software Toolchain~\cite{plfcc}, a system for proving correctness of C programs that relies on style 1 lock specs for its soundness proof w.r.t. the CompCert compiler~\cite{compcert}, and has recently been extended with support for general ghost state and atomicity proofs. We demonstrate that 1) \textbf{our approach can be used to prove atomic specs for interesting data structures with fine-grained locking}, and 2) \textbf{these proofs require some additional complications that can be avoided with style 2 lock specs}. The key difference is that in style 1 a thread must gain ownership of the lock invariant assertion for a component before acquiring the lock and reading/modifying the component, while in style 2 the locks and components alike can be considered part of a single abstract state that is accessed at each lock access. However, the top-level specifications proved for the data structure are not affected by the complexity of the lock specs, so proofs of clients of the data structure are no more complex than they would be otherwise. To our knowledge, this is the first formal comparison of the two styles of lock specifications, and our technique may be useful in tools like VST where the older style is deeply integrated into the tool.

%In practice, however, the interaction between traditional lock-based separation logic and atomic specifications has been underexplored. In this paper, we demonstrate an approach to proving that lock-based implementations of data structures satisfy logically atomic specifications. The approach has two main components. First, we associate each lock with a piece of ghost state representing the abstract state of the part of the data structure it protects. In a \emph{coarse-grained} implementation, where a single lock protects the entire data structure, this is sufficient (in combination with proof rules for logical atomicity) to prove that operations complete atomically. In a \emph{fine-grained} implementation, however, a lock may only protect a small piece of the data structure (a single node in a tree, for instance), and the overall state of the data structure may be composed of a large number of per-lock abstract states. For these data structures, our second step is to prove lemmas describing how changes to local abstract state propagate to change (or maintain) the global abstract state. This gives us a two-layer verification process: we prove that the code of each function implements some local changes to the abstract state, and separately we prove that these local changes implement a global change.

%We demonstrate our technique by proving the correctness of a binary search tree with hand-over-hand locking. %something about client

%Our specific contributions are:
%\begin{itemize}
%\item To the best of our knowledge, this is the first mechanized
%verification of a concurrent search-based data structure written in a
%real programming language. %should probably clarify what we mean by "real"

%\item We illustrate how to incorporate the CSL in VST, the higher-order
%ghost state in the style of Iris, and the logical atomicity from the
%TaDA logic together to verify the CBST.
%\end{itemize}

%We introduce the background about the verification of concurrent C
%programs in \S\ref{background}, including the tool-chain VST and Iris,
%the concept of ghost states, global invariants, and atomic
%specifications. The thread-safety proofs of operations on CBST are
%first explained in \S\ref{safety}, where we show the \emph{recursive}
%lock pattern for hand-over-hand locking mechanism. The functional
%correctness proofs are presented in \S\ref{correctness}. We detail the
%use of the \textsf{sync\_inv} pattern, the combination of recursive
%lock invariants and ghost states together in the atomic
%specifications, and the proof skeleton of each operation on CBST. The
%related work is discussed in \S\ref{related}. We end with the
%conclusion of our work in \S\ref{conclusion}.

