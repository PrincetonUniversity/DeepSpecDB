\documentclass[runningheads]{llncs}

\bibliographystyle{splncs04}% the mandatory bibstyle

\usepackage{txfonts}
\usepackage[T1]{fontenc}
% T1 fonts will be used to generate the final print and online PDFs,
% so please use T1 fonts in your manuscript whenever possible.
% Other font encondings may result in incorrect characters.
%
\usepackage{graphicx}
% Used for displaying a sample figure. If possible, figure files should
% be included in EPS format.
%
\usepackage{subcaption} %% For complex figures with subfigures/subcaptions
                        %% http://ctan.org/pkg/subcaption
\usepackage{uri}
\usepackage{mathpartir}
\usepackage{semantic}
\usepackage{cases}
\usepackage{hyperref}
% If you use the hyperref package, please uncomment the following two lines
% to display URLs in blue roman font according to Springer's eBook style:
\usepackage{color}
\renewcommand\UrlFont{\color{blue}\rmfamily}
\usepackage{stmaryrd}
\usepackage{listings}
\usepackage{amssymb}
\usepackage{iris-llncs}
%\usepackage{lstlangcoq}
\usepackage[edges]{forest}
\renewcommand{\lstlistingname}{Figure}

\clubpenalty = 10000
\widowpenalty = 10000
\displaywidowpenalty = 10000

\lstset{language=C,basicstyle=\ttfamily,mathescape=true,columns=fullflexible}

\newcommand{\wm}[1]{\textbf{\textcolor{blue}{[ WM: #1]}}}
\newcommand{\LB}[1]{\textbf{\textcolor{red}{[ LB: #1]}}}
\newcommand{\TODO}[1]{\textbf{\textcolor{red}{[ TODO: #1]}}}
%\newcommand{\boxdotright}{\!\mathrel\boxdot\joinrel\rightarrow\!}
\newcommand{\islock}{\boxdotright}
\newcommand{\lockvar}{\islock^{v}}
\newcommand{\isaex}{\!\mathrel\odot\joinrel\rightarrow\!}
\newcommand{\xisaex}[1]{\!\mathrel\odot\joinrel\xrightarrow{#1}\!}
%% \newcommand{\ifthenelse}[3]{\text{if }#1\text{ then }#2\text{ else }#3}
\newcommand{\emp}{\mathsf{emp}}

\newcommand\dboxed[1]{\dbox{\ensuremath{#1}}}
\newcommand{\master}[2]{\ensuremath{\mathrm{Master}_{#1}(#2)}}
\newcommand{\snap}[1]{\ensuremath{\mathrm{Snapshot}(#1)}}
\newcommand{\ghost}[2]{\ensuremath{\dboxed{#1}^{#2}}}
\newcommand{\us}{$\mu$s}
\newcommand{\gnamety}{\ensuremath{\mathsf{gname}}}
\newcommand{\treerep}{\ensuremath{\mathsf{bst\_abs}}}
\newcommand{\nodeboxrep}{\ensuremath{\mathsf{bst\_ref}}}

\newcommand{\nodeghost}[2]{\ensuremath{\mathsf{node\_ghost}_{#1}\ (#2)}}
\newcommand{\myhalf}[2]{\ensuremath{\mathsf{node\_ghost}^{.5}_{#1}\ (#2)}}
\newcommand{\publichalf}[2]{\ensuremath{\mathsf{node\_ghost}^{.5}_{#1}\ (#2)}}

\newcommand{\ignore}[1]{}

\hyphenation{Comp-Cert}

\usepackage[utf8]{inputenc}

\begin{document}

\title{Proving Logical Atomicity using Lock Invariants\thanks{Supported by the National Science Foundation via grant \#1811894.}}
%
%\titlerunning{Abbreviated paper title}
% If the paper title is too long for the running head, you can set
% an abbreviated paper title here
%
\author{Roshan Sharma\inst{1} \and
Shengyi Wang\inst{2}\orcidID{0000-0002-2286-8703} \and
Alexander Oey\inst{3} \and
Anastasiia Evdokimova\inst{4} \and
Lennart Beringer\inst{2}\orcidID{?} \and
William Mansky\inst{4}\orcidID{0000-0002-5351-895X}} %confirm author order

\authorrunning{R. Sharma et al.}

\institute{Amazon, USA \and Princeton University, USA \and Cisco, USA \and University of Illinois Chicago, USA} %do we need a note about prior affiliations?

%\ccsdesc[100]{\textcolor{red}{Replace ccsdesc macro with valid one}} https://dl.acm.org/ccs/ccs_flat.cfm 

%\nolinenumbers %uncomment to disable line numbering

\maketitle

\begin{abstract}
Logical atomicity has been widely accepted as a specification format for data structures in concurrent separation logic. While both lock-free and lock-based data structures have been verified against logically atomic specifications, most of the latter start with atomic specifications for the locks as well. In this paper, we compare this approach with one based on older lock-invariant-based specifications for locks. We show that we can still prove logically atomic specifications for data structures with fine-grained locking using these older specs, but the proofs are significantly more complicated than those that use atomic lock specifications. Our proof technique is implemented in the Verified Software Toolchain, which relies on older lock specifications for its soundness proof, and applied to C implementations of lock-based concurrent data structures.

\keywords{concurrent separation logic, fine-grained locking, logical atomicity, Verified Software Toolchain, Iris}
\end{abstract}

% Contributions:
% verified C implementation of BST -- the first?
% technique for hand-over-hand locking with share accounting -- slight modification of Gotsman's for logics without held predicates that remember the invariant; amounts to converting Locked_I into another share of the lock assertion
% technique for connecting locks with abstract state/atomicity
% VST-style decomposition of proofs into C-to-local-abstract and local-to-global-abstract
% lock-free implementation?

\section{Introduction}
The binary search tree (BST) is a common implementation of an ordered
map, a widely used data structure. The concurrent version of the
ordered map forms the bedrock of many parallel programs. We formally
verified the functional correctness of two versions of fine-grained
concurrent binary search trees (CBST) written in C. One adopts the
hand-over-hand locking technique \cite{bayer1977}, the other is lock
free by using atomic primitives such as compare and swap (CAS). Both
CBST implementations support insert, lookup, and delete operations; %or do we have lock-free without delete?
they also share the ``same'' specifications to some extent in our
verification. All the proof is machine-checked in Coq. %Say something about existing proofs of fine-grained data structures and how we compare.

Our specifications use the logical atomicity introduced in the TaDA
logic \cite{tada}, in the form of \emph{atomic Hoare
triples}. Intuitively, an atomic Hoare triple $ \langle
P \rangle\,\mathbb{C}\, \langle Q \rangle$ means the program
$\mathbb{C}$ ``atomically updates'' from $P$ to $Q$. The program may
actually take multiple steps, but every step before the atomic update
(linearization point) must preserve the assertion $P$. Meanwhile, the
concurrent environment may also update the state before the
linearization point, as long as the states satisfies $P$. The
assertion $Q$ must become true at the linearization point, then the
environment can do whatever it likes. There is no guarantee that $Q$
is still preserved when $\mathbb{C}$ returns. For example, the
specification of our \texttt{insert} operation may be explained as
follows: during the execution of \texttt{insert}, there is
always \emph{some} BST, and at some point the \texttt{insert} will
take a BST $t$, insert a value with a certain key, and then eventually
returns (meanwhile other threads may have further modified the
inserted tree)\footnote{Maybe say what a reasonable $Q$ is and why such a spec is still useful, specifically w.r.t. the situation where one of the other threads' action following the linearization point removes the element just inserted.}.

We employ the Verified Software Toolchain (VST) \cite{plfcc} to verify
the correctness of CBST. Although the concurrent separation logic
(CSL) has been formalized in VST rooted on the work of Hobor et
al.\ \cite{oraclesematic}, we extend it with two descendants of CSL so
as to accomplish the verification. One is the logical atomicity
mentioned above; the other is the higher-order ghost state in the
style of Iris \cite{higherorderghoststate}. The ghost states are used
to contruct both the global invariants and the local state in our
proofs. They will be further discussed in \S\ref{correctness}.

We highlight a few innovative aspects of our verification of CBST:
\begin{description}
\item [Range ghost] We abstract the BST via a pair of values
called \emph{range} which represents the lower-bound and upper-bound
of keys on each node to reason about the BST in the current
settings. We prove that range with the merging operation forms a
partial commutative monoid (PCM) so that it can be encoded in ghost
states.

\item [\textsf{sync\_inv} pattern] It is a particular approach
combining locks with general invariants to solve the dilemma caused by
the fine-grained locking mechanism: we do not have a lock to control
the access of the entire BST nor can we access the state of the BST
via atomic operations.
\end{description}

%% Do we have contributions, and are they sufficient? Range ghost is adapted from flow interfaces. Sync_inv pattern isn't totally ours, though it might be instructive w.r.t. combining traditional and Iris-style CSLs, and it might be worth explaining in more detail. We're working on "actual" code, but missing a few pieces to make our proofs foundational. (but maybe it's worth highlighting the extra work we do to work on actual code) We could promote this as the first working example of Iris+VST. We don't yet have any demonstrations on larger or interesting data structures, and this alone was quite a bit of work. The delete logic is rather interesting. Nothing to say about flow interfaces yet, but maybe soon.


Our specific contributions are:
\begin{itemize}
\item To the best of our knowledge, this is the first mechanized
verification of an concurrent search-based data structure written in a
real programming language. %should probably clarify what we mean by "real"

\item We illustrate how to incorporate the CSL in VST, the higher-order
ghost state in the style of Iris, and the logical atomicity from the
TaDA logic together to verify the CBST.
\end{itemize}

We introduce the background about the verification of concurrent C
programs in \S\ref{background}, including the tool-chain VST and Iris,
the concept of ghost states, global invariants, and atomic
specifications. The thread-safety proofs of operations on CBST are
first explained in \S\ref{safety}, where we show the \emph{recursive}
lock pattern for hand-over-hand locking mechanism. The functional
correctness proofs are presented in \S\ref{correctness}. We detail the
use of the \textsf{sync\_inv} pattern, the combination of recursive
lock invariants and ghost states together in the atomic
specifications, and the proof skeleton of each operation on CBST. The
related work is discussed in \S\ref{related}. We end with the
conclusion of our work in \S\ref{conclusion}.



\section{Background}
\label{background}

\subsection{Logically Atomic Specifications}
\label{atomic}

For many concurrent data structures, the ideal correctness condition is that the data structure behaves the same as a sequential implementation, even when accessed simultaneously by multiple threads. This intuitive condition has been formalized in terms of linearizability (operations appear to take effect in some total order) or atomicity (client threads never observe intermediate states of operations). In separation logic, atomicity can be expressed in the form of \emph{atomic triples}~\cite{tada} of the form \[\langle a.\ P_l\ |\ P_p(a)\rangle\ c\ \langle Q_l\ |\ Q_p(a)\rangle\] where $P_l$ and $Q_l$ are \emph{local} pre- and postconditions similar to an ordinary Hoare triple, and $P_p$ and $Q_p$ are \emph{public} pre- and postconditions, parameterized by an abstract value $a$ of the shared data structure. This triple says that if $P_l$ is true before a call, and the fact that $P_p$ is true for some value of $a$ is held in shared state (e.g. a global invariant), then $P_p$ will continue to be true for some (possibly different) value of $a$ up until the \emph{linearization point} of $c$, at which point $Q_p$ will become true atomically for the same value $a$ (and $Q_l$ will be true after $c$ ends). \LB{This suggests I should interpret the leftmost occurrence of 'a' as a binder whose scope extends all the way to $Q_p(a)$. So maybe the binding should happen outside the left pair of angle brackets?}\wm{In the original TaDA paper they did this with a fancy $\forall$ outside that I can't figure out how to replicate. In Iris, they use the notation I'm using here, even though the scope looks wrong.} \LB{Also, as decorations 'l' and 'p' are not indices but just markers (they don't range over anything) and both 'private' and 'public' start with the letter 'p', maybe replacing $P_l$, $P_p$, $Q_l$, $Q_p$ with say $P$, $Q$, $R$, $S$, or maybe $P$, $P'$, $Q,$, $Q'$ could help. Indeed, the concrete patters below suggest something like $P$, $Q$, $P'$, $Q'$, to emphasize that the two predicates to the left of $\mid$ have something in common, and so do the two predicates to the right of $\mid$.} The specific value of $a$ (the shared abstract state) can vary as other threads modify the data structure, as long as $P_p(a)$ is always maintained. For instance, the specification \[\langle s.\ \mathsf{is\_stack}\ p\ |\ \mathsf{stack}\ s\rangle\ \texttt{push}(v)\ \langle \mathsf{is\_stack}\ p\ |\ \mathsf{stack}\ (v :: s)\rangle\] expresses the fact that the \texttt{push} operation of a concurrent stack correctly implements the behavior of a sequential push, making no visible changes to whatever stack $s$ is in place until, at the linearization point, the current stack $s$ is replaced atomically with $v :: s$. The $\mathsf{stack}$ itself is treated as a publicly owned resource, and can only be accessed and modified atomically by threads holding the $\mathsf{is\_stack}$ assertion.

\subsection{Two Styles of Lock Specification}
Soon after the introduction of concurrent separation logic~\cite{csl}, both Gotsman et al.~\cite{gotsman} and Hobor et al.~\cite{oraclesematic} extended it with support for C-style storable locks, where a lock is a special memory location designated as coordinating access to some other resource. Their lock specifications were essentially equivalent, and of the form:
\[\{ \ell \islock R \}\ \texttt{acquire}(\ell)\ \{ R * \ell \islock R \} \qquad \{ R * \ell \islock R \}\ \texttt{release}(\ell)\ \{ \ell \islock R \}\]
Each lock $\ell$ is associated with a ``lock invariant'' $R$, an arbitrary separation logic predicate representing the protected resource, which is gained by a thread that acquires the lock and must be restored when the lock is released. At the time, there was no formal treatment of atomic operations in CSL, so actual lock implementations were beyond the scope of verification: these specs were taken as axioms, or proved through direct appeal to data-race-free guarantees of specific memory models. In garbage-collected languages that do not deallocate locks, the assertion $\ell \islock R$ can be freely duplicated and shared among threads accessing $R$; if we want to track ownership and eventually deallocate the lock, we can instead split the assertion into fractional shares such that $\ell \islock_{\pi_1} R * \ell \islock_{\pi_2} R = \ell \islock_{\pi_1 + \pi_2} R$, and then allow deallocation only when we own the full share $\ell \islock_1 R$.

Later, the TaDA logic~\cite{tada}, which codified the notion of logical atomicity, presented an alternative set of lock specifications:
\[\langle b.\ (\mathsf{L}(\ell) \land \neg b) \vee (\mathsf{U}(\ell) \land b) \rangle\ \texttt{acquire}(\ell)\ \langle \mathsf{L}(\ell) \land b \rangle \qquad \langle \mathsf{L}(\ell) \rangle\ \texttt{release}(\ell)\ \langle \mathsf{U}(\ell) \rangle\]
An \texttt{acquire} operation may access the lock repeatedly, finding it in either the locked ($\mathsf{L}$) or unlocked ($\mathsf{U}$) state; if it sees the unlocked state ($b$ holds), then it can atomically move to a locked state and return. A release simply moves atomically from the locked state to the unlocked state. These operations make no mention of a protected resource, but because they are atomic, they can interact with any global invariant (``shared region'' in TaDA's terminology): in particular, in combination with the invariant $(\mathsf{U}(\ell) * R) \vee \mathsf{L}(\ell)$, they can be used to derive the lock-invariant-based specs. It is also worth noting that these lock specifications are not axioms: they are proved for a simple spinlock implementation that uses atomic operations (e.g. compare-and-swap), which are the actual primitive operations in TaDA and the logics that follow it. In this style, tracking ownership of a lock is orthogonal to the lock assertion itself; a mechanism such as cancelable invariants~\cite{rustbelt-relaxed} can be used to both make the lock assertion $\mathsf{U}$/$\mathsf{L}$ publicly available and deallocate it once all threads are finished.

\subsection{VST and Iris}

Our proof technique is developed in a combination of two Coq-based separation logic verification systems, the Verified Software Toolchain (VST) and Iris. VST~\cite{plfcc} is a Coq-based system for proving separation logic specifications of C programs. It is proved sound against not only the C semantics of the CompCert compiler, but also the compiler correctness theorem~\cite{cpm}, so that verified programs are guaranteed to behave correctly when compiled and run. VST directly implements the concurrent separation logic of Hobor et al.~\cite{oraclesematic}, and the invariant-style lock specifications are incorporated into its soundness proof at a fundamental level.

Iris~\cite{iris} is a framework that synthesizes ideas from a range of CSLs with the idea that ``ghost state is all you need''. Iris begins with a small core logic including arbitrary higher-order ghost state, and derives features including invariants and logical atomicity from particular instances of ghost state. Iris is designed as a language- and logic-independent framework, and recent versions of VST take advantage of this, incorporating Iris-style ghost state in the foundational model and directly using Iris tactics to reason about invariants and atomicity of C implementations. All proofs described in this paper have been formalized in this VST+Iris setting. Iris has also been used as the basis for proofs with atomic lock specifications, such as the work of Krishna et al.~\cite{templates}.
%What else do we need to say?

\ignore{\subsection{Ghost State and Global Invariants}
We can prove memory safety and race freedom properties of shared-memory concurrent programs, by tracking the transfer of ownership of shared locations between threads that would happen if we ran the program. But, this idea is insufficient to prove that concurrently running threads are successful to accomplish some task, which is also called functional correctness. For instance, we can prove the memory safety properties of the increment example in  \ref{figure1} through the transfer of ownership of shared variable $x$ between threads. But, we can not guarantee that the value of $x$ will be $2$ after all threads complete their execution, which is demonstrated in \ref{figure1} using separation logic assertion.   
\begin{figure}[htb]
\centering
$\texttt{x = 0;}$\\
$\mathit{I_{\pi}(x \mapsto v \land v \geq 0)}\qquad\pi = \pi_1\ .\ \pi_2$\\
$\begin{array}{l || l}
\mathit{I_{\pi_1}(x \mapsto v \land v \geq 0)} & \mathit{I_{\pi_2}(x \mapsto v \land v \geq 0)}\\
\texttt{acquire(l);} & \texttt{acquire(l);}\\
\mathit{x \mapsto v \land I_{\pi_1}(x \mapsto v \land v \geq 0)} & \mathit{x \mapsto v \land I_{\pi_2}(x \mapsto v \land v \geq 0)}\\
\texttt{x++;} & \texttt{x++;}\\
\mathit{x \mapsto v+1 \land I_{\pi_1}(x \mapsto v \land v \geq 0)} & \mathit{x \mapsto v+1 \land I_{\pi_2}(x \mapsto v \land v \geq 0)}\\
\texttt{release(l);} & \texttt{release(l);}\\
\mathit{I_{\pi_1}(x \mapsto v \land v \geq 0)} & \mathit{I_{\pi_2}(x \mapsto v \land v \geq 0)}\\
\end{array}$\\
$\mathit{I_{\pi}(x \mapsto v \land v \geq 0)}$
\caption{The increment example annotated with separation logic assertion}
\label{figure1}
\end{figure}
We must preserve the connection between the value of shared resources and the work performed by each thread. A common approach to achieve this is to use \emph{ghost variables}, an auxiliary state introduced later in the proof to track the local information about each thread. They do not appear in the original program. Program in \ref{figure1} can be verified by creating ghost state, which tracks the latest action performed by a thread, for each thread with initial value of $0$, and update with $1$ after each thread increment the value of \texttt{x}. In VST, we can create \emph{ghost state} for any Coq type in the form of an arbitrary \emph{partial commutative monoid} (PCM), a set with a partially defined binary operation that is as associative as it can, commutative, and has a unit, as long as we can describe what happens when two elements of that type are joined together. VST provides $\mathsf {ghost\_var}$ assertion to create a simple ghost state: $\mathsf{ghost\_var}\ \mathit{sh}\ a\ g$ asserts that $g$ is a \emph{ghost name} ($\gnamety$ in Coq) associated with the value $a$, which may be of any type. We will see different kind of ghost states used to verify binary search tree in the following sections.

The main idea behind the logic from Iris \cite{higherorderghoststate}, a mechnized higher-order concurrent separation logic framework, is the construction of \emph{global invariant} as \emph{ghost state}. \emph{Global Invariant} is an invariant on the global (ghost and physical) state of program. We can open any invariant but must close it again before taking any steps of execution unless those execution are atomic. Thread can use the contents of \emph{global invariant} during atomic operation if it can guarantee that no one will ever see an intermediate state in which invariant does not hold. The rules from Jung et al.\cite{higherorderghoststate} for creating and opening invariants are:

\begin{mathpar}
\inference[\textsf{inv\_alloc}]{}{\triangleright P \vdash \pvs[E] \mathsf{EX}\ i : \mathsf{iname}, \knowInv{i}{P}}

\inference[\textsf{inv\_open}]{i \in E}{\knowInv{i}{P} \vdash \pvs[E][E \setminus i] \triangleright P * (\triangleright P \wand \pvs[E \setminus i][E] \mathsf{emp})}
\end{mathpar}

where invariants are provided in the form of assertion $\knowInv{i}{P}$, and states that $P$ is maintained as an invariant on the global state with name $i$. The operator%%  $\pvs[][]$
is called ``fancy update" operator,that allows to allocate, open, and close the invariants and the later $\triangleright$ operator is used for impredicativity (i.e. $\knowInv{i}{..\knowInv{i}{P}..}$).}

\ignore{\subsection{Atomic Specifications}
\label{atomic}

For many concurrent data structures, the ideal correctness condition is that the data structure behaves the same as a sequential implementation, even when accessed simultaneously by multiple threads. This intuitive condition can be formalized as linearizability (operations appear to take effect in some total order) or atomicity (client threads never observe intermediate states of operations). In separation logic, atomicity can be expressed in the form of \emph{atomic triples}~\cite{tada}, written in the form $\forall a.\ \langle P_l\ |\ P_p(a)\rangle\ c\ \langle Q_l\ |\ Q_p(a)\rangle$, where $P_l$ and $Q_l$ are \emph{private} pre- and postconditions similar to an ordinary Hoare triple, and $P_p$ and $Q_p$ are \emph{public} pre- and postconditions, parameterized by an abstract value $a$ of the shared data structure. Intuitively, $P_l$ and $Q_l$ must be true before and after the call, while $P_p$ must be true for some value of $a$ at every point from the beginning of $c$ until some designated linearization point, at which point $Q_p$ becomes true atomically for the same value $a$. For instance, the specification \[\forall s.\ \langle \mathsf{is\_stack}\ p\ |\ \mathsf{stack}\ s\rangle\ \texttt{push}(v)\ \langle \mathsf{is\_stack}\ p\ |\ \mathsf{stack}\ (v :: s)\rangle\] expresses the fact that the \texttt{push} operation of a concurrent stack correctly implements the behavior of a sequential push, atomically transitioning from some stack $s$ to $v :: s$ at some point during its execution. %maybe something about linking ghost state, which parts are real vs. ghost, accessing invariants around atomic triple
%Maybe say what a reasonable $Q$ is and why such a spec is still useful, specifically w.r.t. the situation where one of the other threads' action following the linearization point removes the element just inserted.

VST has encoded an atomic Hoare triple, and provide the way that matches the notation VST uses for normal specifications. Such specification is called \emph{atomic specification}, and can be formalized in Coq as shown in  \ref{atomic-spec}.  This is how we write pre- and postcondition in VST. 
\begin{figure}[htb]
\centering
\begin{verbatim}
Program Definition insert_spec :=
DECLARE _insert
ATOMIC TYPE W OBJ a INVS Ei Eo
WITH ...
PRE [ ... ]
  PROP (...)
  LOCAL (...)
  SEP (P_l) | (P_p)
POST [ ... ]
  PROP ()
  LOCAL ()
  SEP (Q_l) | (Q_p)
\end{verbatim}
\caption{Atomic Specification in VST}
\label{atomic-spec}
\end{figure}
\texttt{W} is the \textsf{TypeTree} representing the type of arguments passed with \texttt{WITH} clause; \texttt{a} is the abstract state for the triple; \texttt{Ei} and \texttt{Eo} are the sets of invariants names inside and outside the triple. The \texttt{PROP} clause describes things that that are true independent of program state, the \texttt{LOCAL} clause describes the values contained in C local variables, and the \texttt{SEP} clause represents the \emph{separating conjunction} (*) of \emph{spatial predicates}, predicates on some part of the memory. In VST, while proving that a function implements an atomic specification, the precondition will contains an $\mathsf{atomic\_shift}$ assertion with the public pre- and postcondition, and the masks inside it. This atomic shift can be accessed through following two rules:
$$\inference[\textsf{atomic\_commit}]{\forall a, R * P_p\ a \Rrightarrow \mathsf{EX}\ y,\ Q_p\ a\ y * R'\ y}{\textsf{atomic\_shift}(P_p, E_i, E_o, Q_p, Q) * R \Rrightarrow \mathsf{EX}\ y,\ Q\ y * R'\ y}$$
$$\inference[\textsf{atomic\_rollback}]{\forall a, R * P_p\ a \Rrightarrow P_p\ a * R'}{\textsf{atomic\_shift}(P_p, E_i, E_o, Q_p, Q) * R \Rrightarrow \textsf{atomic\_shift}(P_p, E_i, E_o, Q_p, Q) * R'}$$
During commit, we must provide resources $R$ which, in combination with the public precondition $P_p$, allow us to prove the public postcondition $Q_p$. Then, we gain access to an assertion $Q$ required by the postcondition of the function, and leftover resources $R'$. During rollback, we provide resources $R$ which, in combination with the public precondition, allow us to reestablish the public precondition; we then regain the atomic shift back in the proof, as well as leftover resources $R'$. The rollback is specially used to learn some relationship between pieces of information (e.g. ghost state) stored in the public precondition, while the commit is used to prove the public postcondition from the precondition, and obtain an assertion $Q$. To complete the proof of any function's specification, we must always perform commit to obtain $Q$; we can perform any number of rollbacks before that point, but after that point we lose the atomic shift and no access to the public precondition.}

\section{Locks and Atomicity}
\label{atomic}
In this section, we outline our approach to connecting lock invariants and atomic specifications, and contrast it with the atomic-lock-based approach.

Consider a sequential implementation of a binary search tree. It has functions \texttt{insert}, \texttt{lookup}, and \texttt{delete}, proven to satisfy the following specifications:
\begin{mathpar}
\{\mathsf{BST}\:p\ t\}\ \texttt{insert}(p, k, v)\ \{\mathsf{BST}\:p\ (t[k \mapsto v])\}

\{\mathsf{BST}\:p\ t\}\ \texttt{lookup}(p, k)\ \{v.\ \mathsf{BST}\:p\ t \land t(k) = v\}

\{\mathsf{BST}\:p\ t\}\ \texttt{delete}(p, k)\ \{\mathsf{BST}\:p\ (t[k \mapsto \_])\}
\end{mathpar}
where $\mathsf{BST}\:p\ t$ asserts that some abstract binary search tree $t$ is represented in memory starting at location $p$. For instance, $t$ could be an element of a recursively defined tree type in Coq, and $\mathsf{BST}$ could be defined as a recursive predicate that represents each node of $t$ as an object in memory, with fields for key, value, and pointers to left and right children, with the root node located at $p$.

A thread-safe version of this data structure should satisfy atomic specifications that closely correspond to the sequential specs\footnote{We will probably also want functions to create and deallocate trees, but these functions need not be thread-safe, since the tree will not be shared when created and must not be shared when deallocated.}:
\begin{mathpar}
\langle m.\ \nodeboxrep_g\ p\ |\ \treerep_g\ m\rangle\ \texttt{insert}(p, k, v)\ \langle \nodeboxrep_g\ p\ |\ \treerep_g\ (m[k \mapsto v])\rangle

\langle m.\ \nodeboxrep_g\ p\ |\ \treerep_g\ m\rangle\ \texttt{lookup}(p, k)\ \langle v.\ \nodeboxrep_g\ p\ |\ \treerep_g\ m \land m(k) = v\rangle

\langle m.\ \nodeboxrep_g\ p\ |\ \treerep_g\ m\rangle\ \texttt{delete}(p, k)\ \langle \nodeboxrep_g\ p\ |\ \treerep_g\ (m[k \mapsto \_])\rangle
\end{mathpar}
These specifications assert that each operation appears to execute atomically, accessing and updating the state of the data structure without exposing any intermediate states---for instance, a \texttt{lookup} should not find a partially inserted key with a value other than the one passed to its \texttt{insert} call. We split the $\mathsf{BST}$ assertion into two pieces: $\nodeboxrep$ contains the points-to and lock-invariant assertions that a thread needs in order to call a BST function, and $\treerep$ contains the ghost state that describes the overall state of the tree and is treated as an atomically accessed shared resource. They are connected by reference to a fixed but arbitrary ghost state index $g$, which we will generally omit. We change the abstract state of the data structure from a tree $t$ to a key-value map $m$ to reflect the fact that clients cannot observe the internal structure of the tree (which will be useful when we want to rearrange nodes during concurrent access). Our sequential implementation will almost certainly not satisfy these specifications, because its functions relied on being able to access any part of the data structure at any time: the $\mathsf{BST}$ assertion included full ownership of every location in the data structure.

To show that an implementation satisfies these atomic specifications, we must show that 1) the implementation can execute safely without owning any piece of the public assertion $\treerep$, only accessing it atomically and restoring it up until the linearization point, and 2) at some linearization point, the implementation atomically transforms the current map $m$ into one that satisfies the public postcondition. The linearization point may be a (physically or logically) atomic operation, or it may occur between steps of the program (e.g., if the current thread holds a lock on the modified section of the data structure, so other threads cannot distinguish between the original state and the modified state until the lock is released). In the following sections, we discuss the general structure of these proofs for lock-based implementations using each of the two kinds of lock specifications.

\subsection{Atomicity with lock invariants}
The easiest way to synchronize a data structure is to add a lock that each operation acquires at the start and releases at the end. These \emph{coarse-grained} locking implementations are easy to verify with invariant-style lock specs: the lock's invariant can contain the whole data structure assertion (e.g. $\mathsf{BST}\:p\ t$), so that whichever thread is currently executing an operation on the tree at $p$ has full ownership of the data structure. If we only want to prove that the operation is thread-safe, we might choose as our lock invariant $\exists t.\ \mathsf{BST}\ p\ t$, asserting that we gain full ownership of the tree in some state $t$ every time we acquire the lock. On the other hand, if we want to prove specifications that describe changes to the data structure---as in the atomic specification $\langle m.\ \treerep\ m\rangle\ \texttt{insert}(p, k, v)\ \langle \treerep\ (m[k \mapsto v])\rangle$---we need ghost state to track the changes.

Specifically, we can define a piece of ghost state $\mathsf{ghost\_bst}\ t$ that is divided between the lock invariant and the public pre- and postconditions of the functions, connecting the tree in memory to the public abstract state of the data structure. For instance, we might choose a lock invariant $R_{\mathit{cg}} \triangleq \exists t.\ \mathsf{BST}\ p\ t * \mathsf{ghost\_bst}_{.5}\ t$, and then set $\nodeboxrep\ p \triangleq p \islock_\pi R_{\mathit{cg}}$ and $\treerep\ t \triangleq \mathsf{ghost\_bst}_{.5}\ t$. %clarify share
Then we can prove the atomic triples above, where at the linearization point we access $\treerep$, combine the two halves of the ghost state and update them to reflect the new state of the concrete data structure, and then fulfill the atomic postcondition with one half of the ghost state and return the other half to the lock invariant. The atomic postcondition reflects the fact that the abstract state has been accessed or changed, while the lock invariant maintains the connection between the abstract state and the actual data structure in memory. This approach to making changes to a locked data structure visible via ghost state is due to Zhang and Jung's proofs of an atomic syncer in Iris~\cite{atomic-syncer}.%, a higher-order function that wraps any arbitrary functionality in a lock. They proved that if a target function $f$ has a spec of the form $\{R * P\}\ f\ \{R * Q\}$, where $R$ is e.g. the ownership assertion for a data structure, then the locked version $f'$ of $f$ has spec $\langle g. P * \mathsf{ghost}_{.5}\ g\rangle\ f'\ \langle Q * \mathsf{ghost}_{.5}\ g'\rangle$.

\label{atomicity}

In a \emph{fine-grained} implementation, there are multiple locks on different pieces of the data structure, so the ghost state of each lock invariant cannot be the abstract state of the entire data structure. Instead, the ghost state represents the abstract state of the locked section alone, and the abstract state of the data structure is derived from the composition of the states of the locked components. Each locked component $c$ has a piece of ghost state $\mathsf{ghost}\_c$ that is split between the lock invariant and the top-level abstract state. At the end of a critical section for $c$, we atomically access the top-level abstract state, combine the two halves of $\mathsf{ghost}\_c$, update $\mathsf{ghost}\_c$ to reflect any local changes, and then show either that the update to $c$ does not change the top-level state of the data structure, or that the update to $c$ changes the top-level state to one that satisfies the atomic postcondition for the function. For instance, when deleting a node from a binary search tree, we rotate nodes to reconstruct a valid tree; at the end of the critical section for each rotation, we show either that the values in the tree are unchanged (and the tree is still well-ordered), or that we have removed the node to be deleted. %elaborate on two-level structure
As we will see in section~\ref{hand-over-hand}, the lock invariant for each component must also carefully account for the ownership of both that component's lock and the locks of related nodes (e.g., child nodes in a tree).

\ignore{The construction of the lock variants is simple. We begin with a \emph{part-reference} ghost state, a common construction for sharing information between an invariant and the threads that act on it, with elements $\publichalf{a}$ (the current state $a$) and $\myhalf{\pi}{p}$ (partial information about the current state, annotated with a share $\pi$). When the share $\pi$ is 1, we know that $p = a$. (do we want to define this ourselves, or reference it somewhere? note also that this is what Iris calls an authoritative algebra) A lock variant with an assertion $R(a)$ is then defined as %notation
$$\ell \lockvar_\pi R \triangleq \ell \islock \exists a.\ R(a) * \myhalf{\pi}{a}$$ %eliding ghost names here
The share $\pi$ may be set to 1 if the lock's information about $a$ is always fully up to date, or a smaller share if it can become outdated due to operations on other parts of the data structure. Since $a$ is existentially quantified, a thread that changes the value of $a$ forgets it after releasing the lock (as required in a lock invariant), but it must have synchronized with a $\publichalf{a}$ held elsewhere (usually in a global invariant) that will remember the change. Thus, the ghost state acts as a bridge between the component lock invariant and the global abstract state of the data structure. %smaller example before we get into BST?
}

\ignore{From the atomic rules of section~\ref{atomic}, we derive the following rules for accessing and modifying pieces of the global abstract state:
\begin{mathpar}
\inference[\textsf{sync\_commit}]{\forall a.\ R * P_p\ a \Rrightarrow \exists x_1.\ \mathsf{public\_half}(g, x_1) * \exists x_0',\,x_1'.\ (x_0, x_1) \leadsto (x_0', x_1')\ \land \\(\mathsf{my\_half}(g, \mathit{sh}, x_0') * \mathsf{public\_half}(g, x_1') \Rrightarrow \exists y.\ Q_p\ a\ y * R'\ y)}{\textsf{atomic\_shift}(P_p, E_i, E_o, Q_p, Q) * \textsf{my\_half}(g,sh,x_0) * R \Rrightarrow \mathsf{EX}\ y,\ Q\ y * R'\ y}

\inference[\textsf{sync\_rollback}]{\forall a.\ R * P_p\ a \Rrightarrow \exists x_1.\ \mathsf{public\_half}(g, x_1) * (\mathsf{public\_half}(g, x_1) \Rrightarrow P_p\ a * R')}{\begin{array}{c}\textsf{atomic\_shift}(P_p, E_i, E_o, Q_p, Q) * \textsf{my\_half}(g,sh,x_0) * R  \Rrightarrow \\\textsf{atomic\_shift}(P_p, E_i, E_o, Q_p, Q) * \textsf{my\_half}(g,sh,x_0) * R'\end{array}}
\end{mathpar}
A thread that holds $\myhalf{}{a}$ (i.e., is in the critical section of a lock with a lock variant) may interact with the abstract state $P_p$ held in an $\textsf{atomic\_shift}$, as long as the corresponding $\publichalf{}$ is part of the abstract state. It may either leave both halves as is and gain updated resources $R'$ (including, e.g., a snapshot of the current abstract state), or update both halves to a new state that satisfies the public postcondition $Q_p$, executing the linearization point and obtaining the final postcondition $Q$. Typically we will use one of these rules at the end of each critical section, to show that the changes made in the critical section either are invisible to other threads (\textsf{sync\_rollback}) or have fulfilled the linearization point (\textsf{sync\_commit}). We can also derive simpler rules for special cases, such as a version of \textsf{sync\_commit} for read-only operations that do not change the state of any component:

$$\inference[\textsf{sync\_commit\_same}]{\forall a.\ R * P_p\ a \Rrightarrow \exists x_1.\ \mathsf{public\_half}(g, x_1) * (\mathsf{my\_half}(g, \mathit{sh}, x_0) * \mathsf{public\_half}(g, x_1) \Rrightarrow \exists y.\ Q_p\ a\ y * R'\ y)}{\textsf{atomic\_shift}(P_p, E_i, E_o, Q_p, Q) * \textsf{my\_half}(g,sh,x_0) * R \Rrightarrow \mathsf{EX}\ y,\ Q\ y * R'\ y}$$
}

\subsection{Atomicity with atomic locks}
Proving atomic specifications for a data structure using atomic specifications for locks is more immediate. As an example, consider Krishna et al.'s verification of a hand-over-hand locking pattern for search structures~\cite{templates}. The lock specification used in the paper is \[\langle \mathsf{inFP}(n)\ |\ \mathsf{CSS}(r, C) \rangle\ \mathtt{lockNode}\ n\ \langle \mathsf{CSS}(r, C) * \mathsf{N}(n)\rangle\] where $\mathsf{inFP}(n)$ asserts that the node $n$ is in the data structure, $\mathsf{CSS}$ is the abstract state, and $\mathsf{N}(n)$ is the resources for the node $n$ (including both points-to assertions and ghost state). The abstract state assertion $\mathsf{CSS}$ includes both global ghost state and the composition of per-node state $\mathsf{N}$ for each node that is not currently locked, so that $\mathsf{N}(n)$ can be removed from the abstract state when $n$'s lock is acquired.

Note that in this case, the private precondition $\mathsf{inFP}(n)$ is merely the knowledge that $n$ is in the data structure, and does not involve any ownership---in particular, a caller does not need a share of the lock for the root node. A thread can acquire the lock for any node $n$ in the data structure by atomically accessing $\mathsf{CSS}$, extracting the sub-assertion corresponding to $n$, and switching it to the locked state. More generally, in this approach there is no need to track ownership of the locks of individual nodes, which as we will see in section~\ref{selflock} is a source of considerable complexity in lock-invariant-based proofs.

\section{A Verified Binary Search Tree with Hand-over-Hand Locking}
\label{hand-over-hand}%show code, somewhere, maybe cut-down pseudocode

We demonstrate our approach to proving logically atomic specifications with invariant-based (type 1) lock specs by verifying a binary search tree (BST) with fine-grained locking in VST, using the approach of section~\ref{atomicity} to prove atomic specifications for the BST operations while using invariant-style specs for our locks (as required by VST's soundness proof).
As described in section~\ref{atomic}, we aim to define assertions $\nodeboxrep$ (the per-thread handle to the BST) and $\treerep$ (the shared abstract state) such that the operations of the concurrent BST satisfy atomic triples along the lines of \[\langle m.\ \nodeboxrep\ p\ |\ \treerep\ m\rangle\ \texttt{insert}(p, k, v)\ \langle \nodeboxrep\ p\ |\ \treerep\ (m[k\mapsto v])\rangle\] In this section, we describe the construction of the two predicates and the proofs of the operations for a BST implementation with hand-over-hand locking, using lock-invariant-based specs for our locks.

%look at some of the Iris papers in CPP, etc. for how we should describe the proofs, e.g. section 4 of Formal Verification of a Concurrent Bounded Queue in a Weak Memory Model, Mevel and Jourdan, ICFP 2021

%%%%%%%%%%%%%%%%%%%%%

\ignore{\subsection{BST code}

\begin{figure}[htp]
\begin{lstlisting}[language = C, basicstyle=\small\ttfamily]
typedef struct tree {int key; void *value; struct tree_t *left, *right;} tree;
typedef struct tree_t {tree *t; lock_t *lock;} tree_t;
typedef struct tree_t **treebox;
\end{lstlisting}
\begin{subfigure}[t]{0.5\textwidth}
\begin{lstlisting}[language = C, basicstyle=\small\ttfamily, numbers=left]
void insert (treebox t, int x, void *value) {
  struct tree_t *tgt = *t;
  struct tree *p;
  void *l = tgt->lock;
  acquire(l);
  for(;;) {
    p = tgt->t;
    if (p==NULL) {
      tree_t *p1 = malloc(sizeof *tgt);
      tree_t *p2 = malloc(sizeof *tgt);
      p1->t = NULL;
      p2->t = NULL;
      lock_t *l1 = malloc(sizeof(lock_t));
      makelock(l1);
      p1->lock = l1;
      release2(l1);
      lock_t *l2 = malloc(sizeof(lock_t));
      makelock(l2);
      p2->lock = l2;
      release(l2);
      p = malloc(sizeof *p);
      tgt->t = p;
      p->key=x; p->value=value; p->left=p1; p->right=p2;
      release(l);
      return;
    } else {
      int y = p->key;
      if (x<y){
      	tgt = p->left;
        void *l_old = l;
        l = tgt->lock;
        acquire(l);
        release(l_old);
      } else if (y<x){
        tgt = p->right;
        void *l_old = l;
        l = tgt->lock;
        acquire(l);
        release(l_old);
      } else {
      	p->value=value;
        release(l);
      	return;
      }
    }
  }
} 
\end{lstlisting} 
\end{subfigure}\qquad
\begin{subfigure}[t]{0.4\textwidth}
 \begin{lstlisting}[language = C, basicstyle=\small\ttfamily]
void *lookup (treebox t, int x) {
  struct tree *p; void *v;
  struct tree_t *tgt;
  tgt = *t;
  void *l = tgt->lock;
  acquire(l);
  p = tgt->t;
  while (p != NULL) {
    int y = p->key;
    if (x<y){
      tgt = p->left;
      void *l_old = l;
      l = tgt->lock;
      acquire(l);
      p = tgt->t;
      release(l_old);
    } else if (y<x){
      tgt = p->right;
      void *l_old = l;
      l = tgt->lock;
      acquire(l);
      p = tgt->t;
      release2(l_old);
    } else {
      v = p->value;
      release(l);
      return v;
    }
  }
  release(l);
  return NULL;
}
\end{lstlisting}
\end{subfigure}
\caption{BST with hand-over-hand locking}
\label{bst-conc}
\end{figure}
}

\subsection{Hand-over-Hand Locking}
\label{selflock}

Hand-over-hand locking (also called lock coupling) is a fine-grained locking pattern for traversing a data structure in which we acquire the lock for the next node before releasing the lock for the current node, thus guaranteeing that the link between the two nodes is not removed or rearranged while we traverse it. Hand-over-hand locking is a classic example of a pattern that requires lock invariants that refer to other lock invariants, dating back to the earliest papers on ``predicates in the heap'' in CSL~\cite{gotsman}. The trick is to assign each node in the data structure a lock invariant that includes the lock assertion for its child nodes; for a linked list, for instance, we might write \begin{equation}\label{inv1}R(n) \triangleq n \mapsto (d, n', \ell') * \ell' \islock R(n')\end{equation} where $n'$ is the child of $n$ and $\ell'$ is $n'$'s lock. By acquiring $n$'s lock, we learn the lock assertion for $n'$, and can acquire its lock as needed before releasing $n$.

%make this more concise, and maybe merge it into a later section
This invariant becomes slightly more complicated when we consider \emph{ownership} of locks, as is required if we want to eventually free the lock and return its resources. In this case the lock assertion carries a share $\pi$ (often represented as a rational number in the range $(0, 1]$), where the full share $\pi = 1$ is required to free the lock while any share is sufficient to acquire or release it. We can amend our lock invariant to \begin{equation}R(n) \triangleq n \mapsto (d, n', \ell') * \ell' \islock_\pi R(n')\end{equation} for some fixed share $\pi$, but then, what happens to the share of $\ell'$ when we release the lock on $n$? Starting from a state in which we hold $n$'s lock $\ell$, this gives us:
\[\begin{array}{l}
\{ \ell \islock_\pi R(n) * n \mapsto (d, n', \ell') * \ell' \islock_\pi R(n') \}\\
\texttt{acquire}(\ell');\\
\{ \ell \islock_\pi R(n) * n \mapsto (d, n', \ell') * \ell' \islock_\pi R(n') * n' \mapsto (d, n'', \ell'') * \ell'' \islock_\pi R(n'') \}\\
\texttt{release}(\ell);\\
\{ \ell \islock_\pi R(n) * n' \mapsto (d, n'', \ell'') * \ell'' \islock_\pi R(n'') \}
\end{array}\]
In the call to \texttt{release}, we give up resources to reestablish $\ell$'s lock invariant, including the lock assertion for $\ell'$. As a result, we are left with a share of $\ell$ (the lock for the node we left behind) and no share of $\ell'$ (the lock for the current node).

Both of these problems can be solved by making the lock \emph{recursive}, including a share of $\ell$ in $\ell$'s own lock invariant\footnote{Gotsman et al. instead use a separation assertion $\mathsf{Locked}$ that is always held by the thread that acquires a lock, and is required to free the lock: this is functionally equivalent to an extra share of the lock assertion. \LB{Do you mean some thing like Gotsman uses Locked(l), defined by $Locked(l) = l \mapsto_{\pi_1}R(n)$?}\wm{It's not defined that way---it's a separate assertion---but it could be implemented that way.}}. We split ownership of the lock $\ell$ for a node $n$ into two shares: $\pi_1$ is held by $\ell$ itself, and $\pi_2$ is held by the lock of $n$'s parent. Then we amend definition \ref{inv1} to
\[R(n) \triangleq \ell \islock_{\pi_1} R(n) * n \mapsto (d, n', \ell') * \ell' \islock_{\pi_2} R(n')\]
Now the same sequence of operations gives us:
\[\begin{array}{l}
\{ \ell \islock_{\pi_1} R(n) * n \mapsto (d, n', \ell') * \ell' \islock_{\pi_2} R(n') \}\\
\texttt{acquire}(\ell');\\
\{ \ell \islock_{\pi_1} R(n) * n \mapsto (d, n', \ell') * \ell' \islock_{\pi_2} R(n') * \ell' \islock_{\pi_1} R(n') * n' \mapsto (d, n'', \ell'') * \ell'' \islock_{\pi_2} R(n'') \}\\
\texttt{release}(\ell);\\
\{ \ell' \islock_{\pi_1} R(n') * n' \mapsto (d, n'', \ell'') * \ell'' \islock_{\pi_2} R(n'') \}
\end{array}\]
When we acquire $\ell'$, we hold both $\pi_1$ and $\pi_2$ of its lock assertion; when we release $\ell$, we return share $\pi_1$ of $\ell$ along with share $\pi_2$ of $\ell'$, but retain share $\pi_1$ of $\ell'$, so we can release it later. The mechanism for defining a lock invariant that includes a share of its own lock assertion may vary across separation logics (VST provides a $\mathsf{selflock}$ assertion for exactly this purpose), but as long as such a mechanism exists, this pattern can be used to implement hand-over-hand lock invariants with share accounting, allowing us to traverse a data structure built with this pattern and then reclaim all of its resources once all threads are finished with it.

This pattern is the main source of complexity in our proofs that is not present in proofs with atomic lock specifications. For instance, in the hand-over-hand locking proofs of Krishna et al.~\cite{templates}, the only information a node holds about its children is the fact that they are in the data structure; their lock assertions are held in the global abstract state along with the rest of their resources. This significantly simplifies the specification and proof of the hand-over-hand synchronization.

\subsection{Specification of the hand-over-hand BST}
The C implementation of a node in our BST is:
\begin{lstlisting}[language = C,numbers = none]
typedef struct tree {int key; void *value;
                         struct tree_t *left, *right;} tree;
typedef struct tree_t {tree *t; lock_t *lock;} tree_t;
\end{lstlisting}
Each node (of type \lstinline{tree_t}) has a \lstinline{lock} field that holds the lock protecting the node, and a \lstinline{t} field that is either \lstinline{NULL} (for a leaf, which contains no key or value) or points to a \lstinline{tree} struct containing the node's key, value, and child pointers. To verify operations on this structure, we need to relate a \lstinline{tree_t} in memory to an abstract state (a mathematical map from keys to values), and divide ownership of those resources between the assertions $\nodeboxrep$ (the resources held by each client thread) and $\treerep$ (the abstract state of the data structure). As described in section \ref{atomicity}, our approach is to associate a piece of ghost state with each node in the tree, which will be shared between that node's lock and the abstract state in $\treerep$, while $\nodeboxrep$ will be a a pointer to the lock of the root node. We also need to make sure that our lock invariants support the hand-over-hand pattern.

\subsubsection{Key Ranges}%can cut this down if need be
When we traverse the BST looking for a key $k$, we will reach either a node containing $k$, or an empty node where $k$ would appear if it was in the tree. To prove correctness of this procedure, we need ghost state that tracks where each key ``should appear'' in the current tree. As observed by Krishna et al.~\cite{krishna2017flow}, this can be done by associating each node with a lower and upper bound on the keys appearing in the subtree rooted at that node. We refer to the product of these bounds as a \emph{range}. The range of a node is inherited from its parent, based on the parent's key: if node $n$ has range $(l, r)$ and key $k$, then its left child has range $(l, k)$ and its right child has range $(k, r)$. Figure~\ref{range_bst} shows an example of a BST with each node labeled with its range, starting with $(-\infty, +\infty)$ at the root and propagated to the empty leaf nodes. 

\usetikzlibrary{positioning}

\begin{figure}[htb]
\centering
%% \forestset{
%%    lab/.style={
%%         label={[anchor=west, scale=0.75] north east:#1},
%%     },
%% }
\begin{forest}
for tree={
  calign=fixed edge angles,
  calign angle=65,
  circle, draw, minimum size=3ex, inner sep=1pt, l sep=0, l=0,
%%     s sep=3mm, l sep=0mm
        }
    % my label/.style={
        % label={[anchor=south west] north east:#1},
    % },
    [ 20, tikz={\node[right=0pt of .north east, scale=0.75]  {$(-\infty,\infty)$};}
        [15, tikz={\node[left=0pt of .north west, scale=0.75]  {$(-\infty,20)$};}]
        [30, tikz={\node[right=0pt of .north east, scale=0.75]  {$(20,\infty)$};}
            [25, tikz={\node[left=-2pt of .west, scale=0.75]  {$(20,30)$};}]
            [50, tikz={\node[right=0pt of .north east, scale=0.75]  {$(30,\infty)$};}
                [45, tikz={\node[right=0pt of .south east, scale=0.75]  {$(30,50)$};}
                    [40, tikz={\node[left=0pt of .north west, scale=0.75]  {$(30,45)$};}
                        [35, tikz={\node[left=0pt of .north west, scale=0.75]  {$(30,40)$};}
                            [, tikz={\node[left=0pt of .north west, scale=0.75]  {$(30,35)$};}]
                            [, tikz={\node[right=0pt of .north east, scale=0.75]  {$(35,40)$};}]
                        ]
                        [,no edge, draw=none]
                    ]
                    [,no edge, draw=none]
                ]
                [,no edge, draw=none]
            ]
        ]
    ]
\end{forest}
\caption{A BST with each node labeled with its range; empty children of nodes other than 35 are omitted}
\label{range_bst}
\end{figure}

At the leaves, these ranges partition the space of all keys not in the tree. If we reach a leaf node with range $(a, b)$, then we know that keys between $a$ and $b$ 1) are not currently in the tree and 2) can correctly be inserted at this leaf.  For example, suppose we want to insert a key $38$ into the tree in figure~\ref{range_bst}. The right child of the node with key 35 is a leaf with range $(35, 40)$, so it is guaranteed to be the right place to insert 38.

\subsubsection{Per-Node and Global Ghost State}
Importantly, the fact that the leaf ranges partition the space of possible keys remains true even if operations in other threads restructure the tree during our traversal. Furthermore, operations further up the tree can only \emph{increase} the range of nodes below them, so if we find a node whose range includes our key, we are guaranteed to have found the right place for it. This makes ranges ideal for use as ghost state: range information held by one thread will not be invalidated by other threads, and is sufficient to show correctness of BST operations. More precisely, we identify each node with a ghost name $g$  \LB{we should likely explain why nodes need identifiers in the first place - (the concept of ghost names is a technique/ artifact of Iris but to what extent is it visible at the top level? One wonders whether there's a bijecton between C-nodes in memory and gnames or some other data structure that keeps track of such an association, and how intensional the resulting reasoning style gets (invariance under renaming gnames...) )}\wm{Not sure we want to get into this, but yes, there should be a bijection between pointers to \lstinline{tree_t}s and a certain subset of ghost names.}, and the abstract state of a node in the tree is a pair of its range and its contents, where the contents are $\mathsf{None}$ for a leaf node and $\mathsf{Some}\ (k, v, g_l, g_r)$ for an internal node with key $k$, value $v$, and left and right children with identifiers $g_l$ and $g_r$. The predicate $\nodeghost{g}{(i, j), n}$ asserts that the node with identifier $g$ has range $(i, j)$ and contents $n$, and will be divided between the lock invariant of the node and the abstract state assertion $\treerep$.

The $\treerep$ predicate describes the conditions under which an abstract map $m$ is implemented by the ghost state of a tree. Intuitively, we should be able to gather the per-node ghost states into an abstract tree $t$ that implements the map $m$, in that each key-value pair in $m$ appears in one of the nodes of $t$. We formalize this by first defining the collection of per-node ghost state fragments needed to form an abstract tree:
\[\begin{array}{r l}
\mathsf{public}&\mathsf{\_halves}(t, g, (i, j)) \triangleq \mathsf{match}\ t\ \mathsf{with}\\
& |\ \mathsf{Leaf} \Rightarrow \publichalf{g}{(i, j), \mathsf{None}}\\
& |\ \mathsf{Node}\ (k, v, t_l, g_l, t_r, g_r) \Rightarrow \publichalf{g}{(i, j), \mathsf{Some}\ (k, v, g_l, g_r)}\ * \\&\qquad\mathsf{public\_halves}(t_l, g_l, (i, k)) * \mathsf{public\_halves}(t_r, g_r, (k, j))
\end{array}\]
We recursively walk through the tree, starting from a node $g$ with range $(i, j)$, and collect the $\mathsf{node\_ghost}$ assertions of each node into a single assertion. We compute the range of each node's children as described above: if the key in the current node is $k$, then the left child has range $(i, k)$ and the right child has range $(k, j)$.

This collection of per-node ghost state describes the contents and structure of the tree, but a thread that acquires a node's lock also needs to know that the node it contains actually appears in the tree $t$. For this purpose, we define ghost state assertions $\mathsf{ghost\_nodes}(S)$ and $\mathsf{in\_tree}\ g$, where $S$ is a set of identifiers and $g$ is an identifier, such that $\mathsf{in\_tree}\ g * \mathsf{ghost\_nodes}(S)$ implies that $g \in S$. Then our final definition of $\treerep$ is:
\[\begin{array}{r l}\treerep_g\ m \triangleq &\exists t.\ t \textrm{ implements } m \land \mathsf{public\_halves}(t, g, (-\infty, +\infty))\ * \\&\mathsf{ghost\_nodes}(\mathrm{ids}(t))\end{array}\]
where $\mathrm{ids}(t)$ is the set of all node identifiers in the abstract tree $t$, and $t$ implements $m$ iff the key-value pairs in $t$ are exactly those in $m$. The range of the root node (named $g$) is $(-\infty, +\infty)$, and the $\mathsf{ghost\_nodes}$ assertion guarantees that if any thread or lock holds an assertion $\mathsf{in\_tree}\ g_i$, then $g_i$ is a node in $t$. Thanks to the existential quantification over $t$, we can rearrange its nodes at any time, as long as it remains a well-formed tree with the same set of node IDs and key-value pairs. We will take advantage of this fact when verifying the \lstinline{delete} function.

Next, we connect the per-node locks of our C implementation with the global abstract state of $\treerep$ by associating each node's lock with a lock invariant assertion $\ell \islock R$, where $R$ includes both the concrete points-to assertion for the node and the corresponding abstract $\mathsf{node\_ghost}$. The lock invariant for a node at location $p$ needs to map the abstract contents $c$ to the values present in memory in the \lstinline{t} field of $p$:
\[\begin{array}{l}
\mathsf{match}\ c\ \mathsf{with}\\
|\ \mathsf{None} \Rightarrow p.\texttt{t} = \texttt{NULL}\\
|\ \mathsf{Some}\ (k, v, g_l, g_r) \Rightarrow \exists p_l, p_r.\ p.\texttt{t} \mapsto (k, v, p_l, p_r)
\end{array}\]
We then combine this with the hand-over-hand invariant pattern of section~\ref{selflock}, giving us a final lock invariant of:
\[\begin{array}{r l}
\mathsf{node\_inv}_g(p) \triangleq \ &\exists i\,j\ c.\ \myhalf{g}{(i, j), c} * p.\texttt{lock} \islock_{\pi_1} \mathsf{node\_inv}_g * \\&\mathsf{match}\ c\ \mathsf{with}\\
&|\ \mathsf{None} \Rightarrow p.\texttt{t} = \texttt{NULL}\\
&|\ \mathsf{Some}\ (k, v, g_l, g_r) \Rightarrow k \in (i, j) \land \exists p_l, p_r.\ p.\texttt{t} \mapsto (k, v, p_l, p_r)\ * \\&\qquad p_l.\texttt{lock} \islock_{\pi_2} \mathsf{node\_inv}_{g_l} * p_r.\texttt{lock} \islock_{\pi_2} \mathsf{node\_inv}_{g_r}
\end{array}\]
The hand-over-hand pattern allows us to acquire $p_l.\texttt{lock}$ or $p_r.\texttt{lock}$ and then release $p.\texttt{lock}$ without losing any shares of the lock assertions.

Now we have everything we need to specify the tree operations\LB{This is the second time these specs are shown\ldots}:
\begin{mathpar}
\langle m.\,\nodeboxrep_g\,p\ |\ \treerep_g\,m\rangle\ \texttt{insert}(p, k, v)\ \langle \nodeboxrep_g\,p\ |\ \treerep_g\,(\mathrm{insert}\ m\ k\ v)\rangle

\langle m.\,\nodeboxrep_g\,p\ |\ \treerep_g\,m\rangle\ \texttt{lookup}(p, k)\ \langle v.\ \nodeboxrep_g\,p\ |\ \treerep_g\,m \land \mathrm{lookup}\,m\ k = v\rangle

\langle m.\,\nodeboxrep_g\,p\ |\ \treerep_g\,m\rangle\ \texttt{delete}(p, k)\ \langle \nodeboxrep_g\,p\ |\ \treerep_g\,(\mathrm{delete}\ m\ k)\rangle
\end{mathpar}
where $\treerep_g\,m$ asserts that global ghost state of the tree implements a map $m$, as described above, and $\nodeboxrep$ (the client's handle to the data structure) is simply a reference to the lock for the root node:
\[\nodeboxrep_g\,b \triangleq \exists p.\ b \mapsto p * p.\texttt{lock} \islock \mathsf{node\_inv}_g(p)\]
Next, we will prove that each of our BST functions satisfies its specification.

\subsection{Proofs: Insert and Lookup}
%How much detail do we want here?

\begin{figure}[ht]
\begin{subfigure}[t]{0.6\textwidth}
\begin{lstlisting}[language = C, basicstyle=\small\ttfamily, numbers=left]
void insert(tree_t** t, int k, void *v) {
  tree_t *tgt = *t;
  acquire(tgt->lock);
  while(1) {
    tree* p = tgt->t;
    if (p == NULL) {
      ... // set tgt->t to a new node with k, v
      release(tgt->lock);
      return;
    }
    if (k < p->key){
      void *l_old = tgt->lock;
      tgt = p->left;
      acquire(tgt->lock);
      release(l_old);
    } else if (p->key < k){
      ... // move to p->right instead
    } else {
    	p->value = v;
        release(tgt->lock);
    	return;
    }
  }
} 
\end{lstlisting} 
\end{subfigure}\qquad
\begin{subfigure}[t]{0.4\textwidth}
 \begin{lstlisting}[language = C, basicstyle=\small\ttfamily]
void *lookup(tree_t** t, int k) {
  tree_t *tgt = *t;
  acquire(tgt->lock);
  while(1) {
    tree *p = tgt->t;
    if (p == NULL) {

      release(tgt->lock);
      return NULL;
    }
    if (k < p->key){
      void *l_old = tgt->lock;
      tgt = p->left;
      acquire(tgt->lock);
      release(l_old);
    } else if (p->key < k){
      ... // move to p->right instead
    } else {
      void* v = p->value;
      release(tgt->lock);
      return v;
    }
  }
}
\end{lstlisting}
\end{subfigure}
\caption{Code outlines for \lstinline{insert} and \lstinline{lookup}}
\label{bst-conc}
\end{figure}

The code for the insert and lookup functions is shown in figure \ref{bst-conc}. The traversal process is the same in each function: we access the current node's key, compare it to the target key \lstinline{k}, and move to the left child if \lstinline{k} is less than the current key and right if it is greater, using hand-over-hand locking. Thus, both loops have the same invariant. At the beginning of each iteration, we are at a node at pointer $p$ with ghost identifier $g$ and hold the lock for that node, giving us the lock invariant $\mathsf{node\_inv}_g(p)$ with some range $(i, j)$. The loop invariant requires that $\texttt{k} \in (i, j)$, that is, the target key always belongs in the subtree of the current node. This guarantees that if we reach an empty leaf, that leaf is the place where key \lstinline{k} should be held. The actual logic of each operation happens when we either reach an empty leaf node (line 7) or find a node with \lstinline{k} as its key (line 22); there are also the linearization points of the two functions.

When an insertion encounters a node with key \lstinline{k}, it simply changes the value at that node to \lstinline{value}. If it reaches a leaf node, it allocates a new node with key \lstinline{k} and value \lstinline{v}, and inserts it at that position.
In both cases, we must show that we can atomically update the abstract state of the tree from $\treerep\ m$ to $\treerep (m[\texttt{k}\mapsto \texttt{v}])$, by showing that the update to the tree in memory corresponds to a local change in the abstract tree that inserts the new key-value pair. The proofs are split into a proof that the C code implements the local update, and a proof that the local update to a ghost-state tree implements the global update. In the first case, the local change is the insertion of $(\texttt{k}, \texttt{v})$ at an empty leaf whose range includes \lstinline{k}, and the addition of two new empty leaves as its children; in the second case, we simply change the value of an internal node whose key is already \lstinline{k}. 
  
\ignore{\begin{figure}[htb]
\centering
\begin{forest}
  for tree={
    calign=fixed edge angles,
    calign angle=50,
    circle, draw, minimum size=3ex, inner sep=1pt,
    l=0, l sep=0.1em }
    [, phantom, for children={fit=band}, s sep+=10mm,
         before drawing tree={%
          tikz+={%
            \node (a) [inner sep=0pt, fit=(!1) (!1ll) (!1ll1111)] {};
            \node (b) [inner sep=0pt, fit=(!l) (!lll111l) (!l1)] {};
            \node [anchor=south] at ($(a.east)!1/2!(b.west)$) {{\large$\Longrightarrow$}};
          },
        },
        [ 20, tikz={\node[right=0pt of .north east, scale=0.75]  {$(-\infty,\infty)$};}
            [15, tikz={\node[left=0pt of .north west, scale=0.75]  {$(-\infty,20)$};}]
            [30, tikz={\node[right=0pt of .north east, scale=0.75]  {$(20,\infty)$};}
                [25, tikz={\node[left=-2pt of .west, scale=0.75]  {$(20,30)$};}]
                [50, tikz={\node[right=0pt of .north east, scale=0.75]  {$(30,\infty)$};}
                    [45, tikz={\node[right=0pt of .south east, scale=0.75]  {$(30,50)$};}
                        [40, tikz={\node[left=0pt of .north west, scale=0.75]  {$(30,45)$};}
                            [35, tikz={\node[left=0pt of .north west, scale=0.75]  {$(30,40)$};}
                                [, tikz={\node[left=0pt of .north west, scale=0.75]  {$(30,35)$};}]
                                [, very thick, tikz={\node[right=0pt of .north east, scale=0.75]  {$(35,40)$};} %weirdly lopsided
                                ]
                            ]
                            [,no edge, draw=none]
                        ]
                        [,no edge, draw=none]
                    ]
                    [,no edge, draw=none]
                ]
            ]
        ]
        [ 20, tikz={\node[right=0pt of .north east, scale=0.75]  {$(-\infty,\infty)$};}
            [15, tikz={\node[left=0pt of .north west, scale=0.75]  {$(-\infty,20)$};}]
            [30, tikz={\node[right=0pt of .north east, scale=0.75]  {$(20,\infty)$};}
                [25, tikz={\node[left=-2pt of .west, scale=0.75]  {$(20,30)$};}]
                [50, tikz={\node[right=0pt of .north east, scale=0.75]  {$(30,\infty)$};}
                    [45, tikz={\node[right=0pt of .south east, scale=0.75]  {$(30,50)$};}
                        [40, tikz={\node[left=0pt of .north west, scale=0.75]  {$(30,45)$};}
                            [35, tikz={\node[left=0pt of .north west, scale=0.75]  {$(30,40)$};}
                                [, tikz={\node[left=0pt of .north west, scale=0.75]  {$(30,35)$};}]
                                [38, very thick, tikz={\node[right=0pt of .north east, scale=0.75]  {$(35,40)$};}
                                    []
                                    []
                                ]
                            ]
                            [,no edge, draw=none]
                        ]
                        [,no edge, draw=none]
                    ]
                    [,no edge, draw=none]
                ]
            ]
        ]
    ]
\end{forest}
\caption{Adding a new key (38) at a leaf}
\label{extract_insert}
\end{figure}}

\ignore{\begin{figure}[htb]
\centering
\begin{forest}
for tree={
    grow=south,
    circle, draw, minimum size=3ex, inner sep=1pt,
    s sep=3mm
        }
    [, phantom, for children={fit=band}, s sep+=20mm,
         before drawing tree={%
          tikz+={%
            \node (a) [inner sep=0pt, fit=(!1) (!1L) (!1F)] {};
            \node (b) [inner sep=0pt, fit=(!l) (!lL) (!lF)] {};
            \node [anchor=south] at ($(a.east)!1/2!(b.west)$) {$\Longrightarrow\quad$};
          },
        },
        [ 20, tikz={\node[right=0pt of .north east, scale=0.75]  {$(-\infty,\infty)$};}
            [15, tikz={\node[left=0pt of .north west, scale=0.75]  {$(-\infty,20)$};}]
            [30, tikz={\node[right=0pt of .north east, scale=0.75]  {$(20,\infty)$};}
                [25, tikz={\node[left=0pt of .north west, scale=0.75]  {$(20,30)$};}]
                [50, tikz={\node[right=0pt of .north east, scale=0.75]  {$(30,\infty)$};}
                    [45, tikz={\node[left=0pt of .north west, scale=0.75]  {$(30,50)$};}
                        [40, tikz={\node[left=0pt of .north west, scale=0.75]  {$(30,45)$};}
                            [35, tikz={\node[left=0pt of .north west, scale=0.75]  {$(30,40)$};}
                                [, tikz={\node[left=0pt of .north west, scale=0.75]  {$(30,35)$};}]
                                [, tikz={\node[right=0pt of .north east, scale=0.75]  {$(35,40)$}; \node[left=0pt of .west, scale=0.35] {$\mathrm{g\_current}$};}
                                ]
                            ]
                            [$g_2$, draw=none]
                        ]
                        [,no edge, draw=none]
                    ]
                    [,no edge, draw=none]
                ]
            ]
        ]
        [ 20, tikz={\node[right=0pt of .north east, scale=0.75]  {$(-\infty,\infty)$};}
            [15, tikz={\node[left=0pt of .north west, scale=0.75]  {$(-\infty,20)$};}]
            [30, tikz={\node[right=0pt of .north east, scale=0.75]  {$(20,\infty)$};}
                [25, tikz={\node[left=0pt of .north west, scale=0.75]  {$(20,30)$};}]
                [50, tikz={\node[right=0pt of .north east, scale=0.75]  {$(30,\infty)$};}
                    [45, tikz={\node[left=0pt of .north west, scale=0.75]  {$(30,50)$};}
                        [40, tikz={\node[left=0pt of .north west, scale=0.75]  {$(30,45)$};}
                            [35, tikz={\node[left=0pt of .north west, scale=0.75]  {$(30,40)$};}
                                [, tikz={\node[left=0pt of .north west, scale=0.75]  {$(30,35)$};}]
                                [, tikz={\node[right=0pt of .north east, scale=0.75]  {$(35,40)$}; \node[left=0pt of .west, scale=0.35] {$\mathrm{g\_current}$};}
                                ]
                            ]
                            [$g_2$, draw=none]
                        ]
                        [,no edge, draw=none]
                    ]
                    [,no edge, draw=none]
                ]
            ]
        ]
    ]
\end{forest}
\caption{Updating an existing key with a new value}
\label{extract_insert2}
\end{figure}}


%\paragraph{Changing the Global State} describe ``extract lemmas'' here---how much detail?

The proof of \lstinline{lookup} is similar, but simpler. We must show that if the key is found in the tree, then the same key-value pair exists in the abstract state, and that if the key is not found then it is not present in the abstract state. The former follows from the lock invariant $\mathsf{node\_inv}$, which guarantees that the key and value in memory match the key and value in the ghost node; the latter follows from the ranges, since if the abstract tree contains a leaf node with range $(i, j)$ such that $\texttt{k} \in (i, j)$, it cannot contain a node with key \lstinline{k}. In either case, the abstract state remains unchanged.

\ignore{\subsubsection{Lookup}

The code for the $\mathsf{lookup}$ method is shown in figure
\ref{lookupproof}. It takes the location of the root pointer and a key
as the arguments. A thread spans the tree to find a given key using
hand-over-hand locking mechanism. Once a thread finds the key in the
tree, it gets the value associated with that key, releases the current
node's lock, and returns the value.
The atomic specification for $\mathsf{lookup}$ is
\[\langle m.\,\nodeboxrep\ p\ |\ \treerep\ m\rangle\ \texttt{lookup}(p, k)\ \langle v.\ \nodeboxrep\ p\ |\ \treerep\ m \land \mathrm{lookup}\ m\ k = v\rangle\]
The \texttt{lookup} function should atomically look up the value of the key $k$ in the tree at $p$.

As before, the key to the verification is the invariant for the main loop:
\[\begin{array}{l} \mathsf{lookup\_inv}(b, x, \mathit{g\_root}, \mathit{range},\mathit{info}) \triangleq\ \exists\ \mathit{lock},\ \mathit{g\_current},\ \mathit{np},\ (x\in \mathit{range})\ \land \ \ghost{\mathsf{my\_half}(\mathit{range},\mathit{info})}{\mathit{g\_current}}\\ *\ R\ np\ * \mathsf{lock\_inv}(\mathit{lock},\mathit{lsh2},R')\ *\ \nodeboxrep(b,\mathit{g\_root})\ *\ \mathsf{atomic\_shift} (P_p,E_i,E_o,Q_p,Q) \end{array}\]
where $P_p$ and $Q_p$ are $\treerep(g\_root, t)$ and $(v =
M[k])\ \wedge\ \treerep(g\_root, t)$ respectively. The proof steps for
the $\texttt{lookup}$ verification are similar to the steps
used in $\texttt{insert}$ verification, with a few differences which we
discuss below.

Figure \ref{lookupproof} shows the $\mathsf{lookup}$ function
with separation logic annotations. The proof starts with
$\nodeboxrep$ and $\mathsf{atomic\_shift}$ assertions as the
precondition and follows the same approach as the $\mathsf{insert}$
proof for the verification of loop body. Once we find the target key
(the last $\mathsf{else}$ clause in the code), we need
to confirm that the value present in the tree corresponds with the value we would find if we looked up the key in the current abstract state. We accomplish this by opening the global
invariant encoded in the $\mathsf{atomic\_shift}$, and proving that
lookup satisfies the public postcondition $\mathsf{Q\_p}$ with the help
of the $\mathsf{sync\_commit\_same}$ lemma described in section
\ref{atomicity}. This is the linearization point of the
$\mathsf{lookup}$ operation and must be done before releasing the
current node's lock. When we move into the left or right sub-tree,
we need to establish $\mathsf{lookup\_inv}$ at the end of the $\mathsf{if}$
or $\mathsf{else\ if}$ clause. Here, we need to show that the key we
are searching for is still in the range of our new node, so we use
$\mathsf{sync\_rollback}$ to extract the bound of the left/right
node encoded as the ghost state in the public precondition
($\treerep(M, g, g\_root)$).

A critical difference from $\texttt{insert}$ is that the
$\texttt{lookup}$ method does not change the shape of the tree. With the
help of $\texttt{sync\_commit\_same}$, we need no longer the
heavyweight $\emph{extract}$ lemma but instead a simpler
$\emph{ramif}$ lemma to locate the current node in the global
invariants:
\begin{verbatim}
Lemma ghost_tree_rep_public_half_ramif: forall tg g_root r_root g_in,
Ensembles.In (find_ghost_set tg g_root) g_in -> ghost_tree_rep tg g_root r_root |-- 
EX r: node_info, !! (range_info_in_tree r r_root tg) && (public_half g_in r * 
(public_half g_in r -* ghost_tree_rep tg g_root r_root)).
\end{verbatim}

\begin{figure}[htp]
\begin{subfigure}[t]{1\textwidth}
 \[\left\{\begin{array}{l} \nodeboxrep(b,g\_root)\ *\ \mathsf{atomic\_shift}(P_p,E_i,E_o,Q_p,Q)\end{array}\right\}\]
\begin{lstlisting}[language = C,  numbers = none]
void *lookup (treebox t, int x) {
  struct tree *p; void *v;  struct tree_t *tgt;
  tgt = *t;  void *l = tgt->lock;
  acquire(l);  p = tgt->t;
 \end{lstlisting}  
 $$\left\{\begin{array}{l} \ghost{\mathsf{my\_half}((-\infty,+\infty),info)}{g\_root}\ *\ R\ b\ *\ \mathsf{lock\_inv}(l,lsh2,R')\ *\ \\
 \nodeboxrep(b,g\_root)\ *\ \mathsf{atomic\_shift}(P_p,E_i,E_o,Q_p,Q)\end{array}\right\} \Rrightarrow \left\{\begin{array}{l} lookup\_inv \end{array}\right\}$$ 
  \begin{lstlisting}[language = C, numbers = none]
    while (p!=NULL) {
       \end{lstlisting}   
   $$\left\{\begin{array}{l} lookup\_inv \end{array}\right\} \triangleq \left\{\begin{array}{l}(x\in range)\land \ghost{\mathsf{my\_half}(range,info)}{g\_current}*\ R\ np\ *\\\mathsf{lock\_inv}(lock,lsh2,R')\ *\ \nodeboxrep(b,g\_root)\ *\ \mathsf{atomic\_shift}(P_p,E_i,E_o,Q_p,Q)\end{array}\right\}$$
      \begin{lstlisting}[language = C,  numbers = none]
    if (x<y){
      tgt=p->left;
      ....
    }else if (y<x){
      tgt=p->right;
     ....
    }else {
    v = p->value;
           \end{lstlisting} 
  $$\left\{\begin{array}{l} \ghost{\mathsf{my\_half}(range,info)}{g\_current}*\ R\ np\ *\mathsf{lock\_inv}(lock,lsh2,R')\ *\ \\\nodeboxrep(b,g\_root)\ *\ \mathsf{atomic\_shift}(P_p,E_i,E_o,Q_p,Q)\end{array}\right\} \Rrightarrow{\textbf{sync\_commit\_same}}$$
$$\left\{\begin{array}{l} \ghost{\mathsf{my\_half}(range,info)}{g\_current}*\ R\ np\ *\mathsf{lock\_inv}(lock,lsh2,R')\ *\ \nodeboxrep(b,g\_root)\ *\ Q\end{array}\right\}$$
        \begin{lstlisting}[language = C,  numbers = none]
      release2(l);
         \end{lstlisting}
       $$\left\{\begin{array}{l} \nodeboxrep(b,g\_root)\ *\ Q\end{array}\right\}$$
         \begin{lstlisting}[language = C, numbers = none]
       return v;} }
  release2(l);  return NULL;  }
 \end{lstlisting} 
\end{subfigure}
\caption{The $\texttt{lookup}$ function annotated with separation logic specification}
\label{lookupproof}
\end{figure} }

\subsection{Delete}

\begin{figure}[htp]
\begin{subfigure}[t]{0.6\textwidth}
\begin{lstlisting}[language = C, basicstyle=\small\ttfamily, numbers=left]
void pushdown_left (tree_t *tgt) {
  while(1) {
    tree *p = tgt->t;
    tree_t *rc = p->right;
    void *lq = rc->lock;
    acquire(rc->lock);
    tree *q = rc->t;
    if (q == NULL) {
      tree_t *lc = p->left;
      ... // move lc->t to tgt; deallocate p, lc, and rc
      release(tgt->lock);
      return;
    } else {
      turn_left(tgt, rc);
      tgt = q->left;
      release(tgt->lock);
    }
  }
}
\end{lstlisting} 
\end{subfigure}\qquad
\begin{subfigure}[t]{0.4\textwidth}
 \begin{lstlisting}[language = C, basicstyle=\small\ttfamily]

\end{lstlisting}
\end{subfigure}
\caption{Code outline for \lstinline{pushdown_left}}
\label{pushdown-left}
\end{figure}

The \lstinline{delete} function uses the same traversal logic as \lstinline{insert} and \lstinline{lookup}, but when it finds the node with the target key, it calls a helper function \lstinline{pushdown_left}, outlined in figure~\ref{pushdown-left}. This function rearranges the nodes in the tree, moving the node to be deleted down through the tree until its right child is empty while maintaining the binary search property. This is accomplished by repeatedly rotating sets of three nodes: the target node's right child \lstinline{rc} is moved up to become its parent, with the target node on the left and the right child's right child moved under the target node, as illustrated in figure~\ref{delete-diagram}.
Once the node to be deleted has no right child, its contents are replaced with those of its left child, and its children are removed from the tree.

The \lstinline{pushdown_left} function changes the structure of the tree in two ways. First, the rotation performed by \lstinline{turn_left} (line 14) changes the ranges of the node to be deleted and its right child (e.g., nodes 40 and 55 in the first step of figure~\ref{delete-diagram}); this is a local change, as the ranges of all other nodes (even the children of the rearranged nodes) are preserved. This is not a linearization point, so the rotation must be considered not to change the abstract state of the tree. We accounted for this by quantifying over the abstract tree $t$ in the definition of $\treerep$; rearranging nodes without changing the key-value map still implements the same abstract state, and other threads cannot assume that the tree structure they have observed is still the current structure of the tree. Second, the final removal of the node (line 10) increases the ranges of the subtree below it; for example, in the last step of figure~\ref{delete-diagram}, the range of node 35 changes from $(30, 40)$ to $(30, 50)$ when the node is removed. This is the linearization point of the function. The change occurs while \lstinline{pushdown_left} holds only the lock of the deleted node, so the algebra for range ghost state must allow a node's range to increase at any time, even when the node's lock is held. Fortunately, \lstinline{insert} and \lstinline{lookup} only use ranges to assert that the target key is in the range of the current node, which is still true if the range increases.

\usetikzlibrary{positioning}

\begin{figure}[htb]
\centering
\forestset{
   lab/.style={
        label={[anchor=west, scale=0.75] north east:#1},
    },
}
\begin{forest}
for tree={
    grow=south,
    circle, draw, minimum size=3ex, inner sep=1pt,
    s sep=3mm, l sep=0
        }
    % my label/.style={
        % label={[anchor=south west] north east:#1},
    % },
    [, phantom, for children={fit=band}, s sep+=20mm,
         before drawing tree={%
          tikz+={%
            \node (a) [inner sep=0pt, fit=(!1) (!1ll) (!1l1ll)] {};
            \node (c) [inner sep=0pt, fit=(!l) (!ll) (!ll111)] {};
            \node (b) [anchor=south] at ($(a.east)!1/2!(c.west)$) {};
            \node [anchor=south] at ($(a.east)!2/5!(b.west)$) {$\Longrightarrow$};
            \node [anchor=south] at ($(b.east)!3/5!(c.west)$) {$\Longrightarrow$};
          },
        },
        [ 30, tikz={\node[right=0pt of .north east, scale=0.75]  {$(-\infty,\infty)$};}
            [25, tikz={\node[left=0pt of .north west, scale=0.75]  {$(-\infty,30)$};}]
            [75, tikz={\node[right=0pt of .north east, scale=0.75]  {$(30,\infty)$};}
                [40, very thick, tikz={\node[left=0pt of .north west, scale=0.75]  {$(30,75)$};}
                    [35, tikz={\node[left=0pt of .north west, scale=0.75]  {$(30,40)$};}]
                    [55, very thick, tikz={\node[right=0pt of .north east, scale=0.75]  {$(40,75)$};}
                        [50, tikz={\node[left=0pt of .north west, scale=0.75]  {$(40,55)$};}]
                        [60, very thick, tikz={\node[right=0pt of .north east, scale=0.75]  {$(55,75)$};}]
                    ]
                ]
                [80, tikz={\node[right=0pt of .north east, scale=0.75]  {$(75,\infty)$};}]
            ]
        ]
        [ 30, tikz={\node[right=0pt of .north east, scale=0.75]  {$(-\infty,\infty)$};}
            [25, tikz={\node[left=0pt of .north west, scale=0.75]  {$(-\infty,30)$};}]
            [75, tikz={\node[right=0pt of .north east, scale=0.75]  {$(30,\infty)$};}
                [55, tikz={\node[left=0pt of .north west, scale=0.75]  {$(30,75)$};}
                    [40, very thick, tikz={\node[left=0pt of .north west, scale=0.75]  {$(30,55)$};}
                        [35, tikz={\node[left=0pt of .north west, scale=0.75]  {$(30,40)$};}]
                        [50, very thick, tikz={\node[right=0pt of .north east, scale=0.75]  {$(40,55)$};}]
                    ]
                    [60, tikz={\node[right=0pt of .north east, scale=0.75]  {$(55,75)$};}]
                ]
                [80, tikz={\node[right=0pt of .north east, scale=0.75]  {$(75,\infty)$};}]
            ]
        ]
        [ 30, tikz={\node[right=0pt of .north east, scale=0.75]  {$(-\infty,\infty)$};}
            [25, tikz={\node[left=0pt of .north west, scale=0.75]  {$(-\infty,30)$};}]
            [75, tikz={\node[right=0pt of .north east, scale=0.75]  {$(30,\infty)$};}
                [55, tikz={\node[left=0pt of .north west, scale=0.75]  {$(30,75)$};}
                    [50, tikz={\node[left=0pt of .north west, scale=0.75]  {$(30,55)$};}
                        [40, very thick, tikz={\node[left=1.8pt of .east, cross out, scale=2, draw]{}; \node[left=0pt of .north west, scale=0.75]  {$(30,50)$};}
                            [35, tikz={\node[left=0pt of .north west, scale=0.75]  {$(30,35)$};}]
                            [,no edge, draw=none]
                        ]
                        [,no edge, draw=none]
                    ]
                    [60, tikz={\node[right=0pt of .north east, scale=0.75]  {$(55,75)$};}]
                ]
                [80, tikz={\node[right=0pt of .north east, scale=0.75]  {$(75,\infty)$};}]
            ]
        ]
        \ignore{[ 30, tikz={\node[right=0pt of .north east, scale=0.75]  {$(-\infty,\infty)$};}
            [25, tikz={\node[left=0pt of .north west, scale=0.75]  {$(-\infty,30)$};}]
            [75, tikz={\node[right=0pt of .north east, scale=0.75]  {$(30,\infty)$};}
                [55, tikz={\node[left=0pt of .north west, scale=0.75]  {$(30,75)$};}
                    [50, tikz={\node[left=0pt of .north west, scale=0.75]  {$(30,55)$};}
                        [35, tikz={\node[left=0pt of .north west, scale=0.75]  {$(30,40)$};}]
                        [,no edge, draw=none]
                    ]
                    [60, tikz={\node[right=0pt of .north east, scale=0.75]  {$(55,75)$};}]
                ]
                [80, tikz={\node[right=0pt of .north east, scale=0.75]  {$(75,\infty)$};}]
            ]
        ]}
    ]
\end{forest}
\caption{Deleting node 40 from the tree via \lstinline{pushdown_left}}
\label{delete-diagram}
\end{figure}

\ignore{\subsubsection{delete}

Similar with insert, delete starts by acquiring the root node lock, so that the thread
can access the information inside the lock invariant of the root node. Then, we prove
the while loop by showing that the precondition satisfies delete\_inv and the 
loop body preserves the loop invariant. The first two cases are similar to insert and
lookup in which the thread finds the target node, traversing by hand-over-hand locking.
Once the target node is located, the thread will call pushdown\_left and return after.
Using the specification of pushdown\_left, we show that the post-condition of 
pushdown\_left implies the loop invariant. Note that before going into pushdown\_left,
the thread is holding the lock to the target node and will return to the delete
method without holding any.

\subsubsection{pushdown\_left}

The proof to pushdown\_left is the key step of proving the delete method. 
First, we show that the precondition of pushdown\_left satisfies the precondition
of the loop invariant, which includes the assertion that the thread is holding the 
lock to the deleted node (i.e. the information inside the lock invariant is accessible)
and an atomic\_shift. Then, the thread accesses that information to acquire a second 
lock to the right child of the deleted node. Due to the hand-over-hand locking 
mechanism this will not cause a deadlock since the thread has already acquired the 
lock to the deleted node. Next, we prove that the action of deleting the node,
with sync\_commit by opening the global invariant and removing the relevant 
ghost names (the target node and its right sentinel node), satisfies the loop
invariant and the postcondition of pushdown\_left.

The second part of pushdown\_left is proving that the turn\_left operation satisfies
the loop invariant. Using a variant of the sync\_rollback rule, we show that the 
turn\_left operation preserves the BST, apart from changes in the ranges of two nodes,
the target node and its right child, and information about the nodes in those two nodes.
To satisfy the structure of the ranges, we only need to prove that the new range for
the target node is encompasses the old range of both nodes. This will imply that
the range of the target node's old parent includes the target node's right child 
new range as the right child becomes the new parent (and the old parent points to 
the right child).

One key step of the proof is proving that two ghost trees are equivalent if the union
of its ghost nodes are equal and the ghost tree is sorted. This allows the global invariant
to be preserved during the turn\_left operation in which no nodes were actually changed
except for the structure of the tree.}


\ignore{\section{Lock-Free BST}
We want to prove that a lock-free BST implementation satisfies the same specification as our hand-over-hand implementation. Unfortunately, provably-correct deletion in a lock-free setting is a research topic in itself (cite?), so we begin with a lock-free BST that only supports insert and lookup. Once again, we want to prove
\begin{mathpar}
\forall t.\ \langle \nodeboxrep\ p\ |\ \treerep\ t\rangle\ \texttt{insert}(p, k, v)\ \langle \nodeboxrep\ p\ |\ \treerep\ (\mathrm{insert}\ t\ k\ v)\rangle

\forall t.\ \langle \nodeboxrep\ p\ |\ \treerep\ t\rangle\ \texttt{lookup}(p, k)\ \langle v.\ \nodeboxrep\ p\ |\ \treerep\ t \land \mathrm{lookup}\ t\ k = v\rangle
\end{mathpar}
though the precise definitions of $\nodeboxrep$ and $\treerep$ will differ. %something about how the fact that the specs are the same really comes across to a client
}

\section{Using the Specifications}

\begin{figure}[htp]
\begin{subfigure}[t]{0.4\textwidth}
\begin{lstlisting}[language = C, basicstyle=\small\ttfamily, numbers=left]
void* update_tree(void* t){
  a = 1;
  b = 2;
  c = 3; 
  d = 4;
  insert(t, 1, &a);
  insert(t, 2, &b);
  insert(t, 1, &c);
  insert(t, 2, &d);
  return NULL;
}
\end{lstlisting} 
\end{subfigure}\qquad
\begin{subfigure}[t]{0.55\textwidth}
 \begin{lstlisting}[language = C, basicstyle=\small\ttfamily]
int retrieve_value(treebox t){
  int y = -1;
  do {
     searchResult = lookup (t,  2);
     if (searchResult != NULL) {
        y = *((int *) searchResult);
     }
  } while (y != 4);
  void* searchResult = lookup (t,  1);
  return *((int *) searchResult);
}
\end{lstlisting}
\end{subfigure}
\caption{Client code}
\label{bst-conc-client}
\end{figure}

To demonstrate that the complexities of our lock-based proofs are hidden from prospective users, we verified a simple client program for the binary search tree, shown in figure \ref{bst-conc-client}. A producer thread running the \lstinline{update_tree} function populates a tree, while a consumer thread running \lstinline{retrieve_value} waits until key 2 is associated with a pointer to the value 4, and then returns the value associated with key 1. This is a data-structure-level version of the message-passing idiom, and demonstrates one of the advantages of atomic specifications: we can use them to synchronize threads and pass ownership of resources via data structure operations. The invariant for the BST \lstinline{t} is a simple state machine with three states, depending on the contents of the abstract map $m$: 1) $m(2) = \texttt{NULL}$, 2) $m(2) \mapsto i$ for some integer $i \neq 4$; and 3) $m(2) \mapsto 4 \land m(1) \mapsto 3$. In this final state, the invariant also holds the resources to be transferred from the producer to the consumer. We also include ghost state to reflect the fact that there is exactly one producer thread (so the producer always knows exactly which values are in the tree) and one consumer thread (so it can retrieve the resources without worrying about interference). The proof of the client uses only the atomic specifications of the BST operations, and is not affected by the style of lock specification.

%The consumer thread repeatedly tries to read the value stored at the key of 2 to see if it is equal to 4. Once 4 can be observed, it returns the value stored with the key of 1. Since all the inserts by the producer thread are performed in a specific order, the consumer thread should always return 3. Since the fact that, once 4 is returned by the lookup of 2, 3 is already stored with the key of 1 was a part of the invariant, the only thing that this part of the proof required was showing that, once the consumer breaks out of the loop, the tree has already reached its final state. Remembering this fact between the two lookup calls – the one that happens in the loop and the one that happens right after – required the addition of the ghost snap, another special ghost variable which could only change its value once. Consumer also owns a unique $\mathsf{reader\_token}$, which is used to replace the resources in the invariant once they can be retrieved. Because this token is unique, as long as the consumer thread owns it, the fact that the resources have not been replaced with it yet can be proven. 

\section{Related Work}
\label{related}
Gotsman et al.~\cite{gotsman}, in the same paper in which they introduced invariant-style lock specs, also verified a linked list with hand-over-hand locking, which became a common example for CSLs that handled fine-grained locking. Among the most recent of these is the work of Krishna et al.~\cite{templates}, who used atomic lock specs to prove correctness of the same data structure. This work updates the work of Gotsman et al. with modern specifications, and directly compares the results to those of Krishna et al.

VeriFast~\cite{verifast}, a separation-logic-based verifier for C and Java programs, also supports lock-based and atomic concurrency~\cite{verifast-conc}, and has been used to verify a hand-over-hand-locking linked list similar to that of Krishna et al. The specifications of both the lock operations and the list itself use a precursor of TaDA's logical atomicity. VeriFast is not a foundationally verified system, but its basic logic is verified against the semantics of a toy language in Coq. 

%Our work is based on the concurrent separation logic of VST in the style of Iris \cite{higherorderghoststate} and the logical atomicity from TaDA logic  \cite{tada}. The notion of the \emph{lower} and \emph{upper} bound is taken from the flow interface paper \cite{krishna2017flow}. Our main technical contributions are the verification of two different implementation of binary search tree (using fine-grained locking and lock-free technique) with respect to the same abstract specification, and first c-level mechanized verification of concurrent search-based data structure (i.e. BST). Most of the other concurrent separation logic (CSL)-based verification works are on toy languages, while VST lets us use the same logic on real C programs.  VeriFast~\cite{verifast-conc} also operates on C code, albeit without a full formal semantics of C.

%Xiong et al. \cite{Xiong2017Abstract} have demonstrated the verification of ConcurrentSkipListMap from java.util.concurrent library using the recent advances in fine-grained concurrency reasoning. Their work is mainly based on the abstract atomicity from TaDA logic, and give two modular specifications for concurrent maps: one specification focus on the entire map structure which is suitable for verifying implementation, and another specification focus on the key-value pairs which appropriate for verifying clients. We use the same idea of atomicity (though implemented in Iris) in our work. 

%TaDA-live: proves atomicity specs of some lock-based data structures incidentally, no mechanization
%search structure templates: lock coupling data structures with atomic specs specifically, really the toy language is the only disadvantage, unclear whether we'd rather replicate them in whole or in part

\section{Conclusion}
\label{conclusion}
While most new CSL proofs of lock-based data structures use TaDA-style atomic specs for their locks, the implications of the switch away from invariant-based specs had not been thoroughly examined. We have now demonstrated that invariant-based specs can still be used to prove logically atomic specs for larger data structures (even for fine-grained implementations) and that this requires somewhat more complex proofs than using atomic lock specs (especially for fine-grained implementations). The old style of lock spec may still be used in systems like VST, where switching would invalidate existing automation or soundness proofs, and in these systems the approach we have outlined can be used to obtain strong correctness properties for fine-grained-locking data structures. When possible, however, we would prefer to use atomic specifications for locks to simplify our proofs. Indeed, we intend to investigate shifting the foundations of VST to use atomic operations and derive atomic specs for locks: while updating the soundness proof will be a considerable effort, using the newer style of lock specification will simplify all future proofs about lock-based data structures.

\subsubsection{Acknowledgements} This work benefited greatly from correspondence with Ralf Jung, who pointed out the atomic syncer example\LB{I suspect the syncer example is a toy data structure, far simpler than a BST?}\wm{It's a generic higher-order function that wraps the argument in a critical section. We're specializing it to a particular data structure here.}  and explained the finer points of logical atomicity in Iris.

%% Bibliography
\bibliography{../sources}
\end{document}
