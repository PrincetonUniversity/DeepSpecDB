%% Commands for TeXCount
%TC:macro \cite [option:text,text]
%TC:macro \citep [option:text,text]
%TC:macro \citet [option:text,text]
%TC:envir table 0 1
%TC:envir table* 0 1
%TC:envir tabular [ignore] word
%TC:envir displaymath 0 word
%TC:envir math 0 word
%TC:envir comment 0 0
%%
%%
%% The first command in your LaTeX source must be the \documentclass command.
\documentclass[sigplan,10pt,anonymous,review]{acmart}\settopmatter{printfolios=true,printccs=false,printacmref=false}
%%
%% \BibTeX command to typeset BibTeX logo in the docs
\AtBeginDocument{%
  \providecommand\BibTeX{{%
    Bib\TeX}}}

\usepackage{listings}
\usepackage{lstlangcoq}
\usepackage{iris}

\lstdefinestyle{customc}{
  belowcaptionskip=1\baselineskip,
  breaklines=true,
  xleftmargin=\parindent,
  language=C,
  showstringspaces=false,
  basicstyle=\ttfamily,
  keywordstyle=\bfseries\color{green!40!black},
  commentstyle=\itshape\color{purple!40!black},
  identifierstyle=\color{blue!50!black},
  stringstyle=\color{orange},
  numbers=left,
}

\lstdefinestyle{customcoq}{
  mathescape=true,
  belowcaptionskip=1\baselineskip,
  breaklines=true,
  xleftmargin=\parindent,
  language=Coq,
  morekeywords={Variant, fun},
  %morekeywords={SOCKAPI,ITREE,data_at,data_at_},
  emph={%
    SOCKAPI,ITree,data_at,data_at_
  },
  emphstyle={\bfseries\color{green!40!red!80}},
  showstringspaces=false,
  basicstyle=\small\ttfamily,
  keywordstyle=\bfseries\color{green!40!black},
  commentstyle=\itshape\color{purple!40!black},
  identifierstyle=\color{violet!80!black},
  stringstyle=\color{orange},
   literate={=>}{{$\Rightarrow$\ }}1
   %{l_}{{l$\hspace{.2ex}$\raisebox{-.46ex}{-}}}{2}
   %{_}{\raisebox{-.46ex}{-}}1
   {-}{{\textsf{-}}}1
   {->}{{$\rightarrow\,$}}1
   {-->}{{$\longrightarrow\,$}}2
   {<->}{$\leftrightarrow$}1
   {<-->}{$\!\longleftrightarrow\,$}1
   {>=}{$\!\ge\,$}1
   {>=>}{$\subtype$}1
   {<=>}{$\Leftrightarrow\,$}2
   {<=}{$\!\le\,$}1
   {forall}{$\forall\,$}1
   {forallb}{forallb}7
   % {exists}{$\exists\,$}1
   {existsb}{existsb}7
   {existsv}{existsv}7
   {\/\\}{{$\wedge$\ }}1
   {|-}{{$\,\vdash\,$}}2
}

\hyphenation{Comp-Cert}
\hyphenation{Certi-KOS}
\hyphenation{Quick-Chick}

\newcommand{\inlinec}[1]{\lstinline[style=customc]{#1}}
\newcommand{\inlinecoq}[1]{\lstinline[style=customcoq,columns=flexible]{#1}}
\newcommand{\ignore}[1]{}
\newcommand{\pk}{\mathit{pk}}
\newcommand{\pv}{\mathit{pv}}
\newcommand{\islock}{\boxdotright}

%% Rights management information.  This information is sent to you
%% when you complete the rights form.  These commands have SAMPLE
%% values in them; it is your responsibility as an author to replace
%% the commands and values with those provided to you when you
%% complete the rights form.
\setcopyright{acmcopyright}
\copyrightyear{2018}
\acmYear{2018}
\acmDOI{XXXXXXX.XXXXXXX}

%% These commands are for a PROCEEDINGS abstract or paper.
\acmConference[Conference acronym 'XX]{Make sure to enter the correct
  conference title from your rights confirmation email}{June 03--05,
  2018}{Woodstock, NY}
\acmPrice{15.00}
\acmISBN{978-1-4503-XXXX-X/18/06}


%%
%% Submission ID.
%% Use this when submitting an article to a sponsored event. You'll
%% receive a unique submission ID from the organizers
%% of the event, and this ID should be used as the parameter to this command.
%%\acmSubmissionID{123-A56-BU3}

%%
%% For managing citations, it is recommended to use bibliography
%% files in BibTeX format.
%%
%% You can then either use BibTeX with the ACM-Reference-Format style,
%% or BibLaTeX with the acmnumeric or acmauthoryear sytles, that include
%% support for advanced citation of software artefact from the
%% biblatex-software package, also separately available on CTAN.
%%
%% Look at the sample-*-biblatex.tex files for templates showcasing
%% the biblatex styles.
%%

%%
%% The majority of ACM publications use numbered citations and
%% references.  The command \citestyle{authoryear} switches to the
%% "author year" style.
%%
%% If you are preparing content for an event
%% sponsored by ACM SIGGRAPH, you must use the "author year" style of
%% citations and references.
%% Uncommenting
%% the next command will enable that style.
%%\citestyle{acmauthoryear}



%%
%% end of the preamble, start of the body of the document source.
\begin{document}

%%
%% The "title" command has an optional parameter,
%% allowing the author to define a "short title" to be used in page headers.
\title{Verifying Concurrent Search Structure Templates for C Programs using VST}

%%
%% The "author" command and its associated commands are used to define
%% the authors and their affiliations.
%% Of note is the shared affiliation of the first two authors, and the
%% "authornote" and "authornotemark" commands
%% used to denote shared contribution to the research.
\author{Duc Than Nguyen}
%\authornote{Both authors contributed equally to this research.}
\email{dnguye96@uic.edu}
\orcid{}
\affiliation{%
  \institution{University of Illinois Chicago}
  \streetaddress{}
  \city{Chicago}
  \state{Illinois}
  \country{USA}
  \postcode{}
}

%%
%% By default, the full list of authors will be used in the page
%% headers. Often, this list is too long, and will overlap
%% other information printed in the page headers. This command allows
%% the author to define a more concise list
%% of authors' names for this purpose.
%\renewcommand{\shortauthors}{Trovato et al.}

%%
%% The abstract is a short summary of the work to be presented in the
%% article.
\begin{abstract}

\end{abstract}

%%
%% The code below is generated by the tool at http://dl.acm.org/ccs.cfm.
%% Please copy and paste the code instead of the example below.
%%

%%
%% Keywords. The author(s) should pick words that accurately describe
%% the work being presented. Separate the keywords with commas.
\keywords{}

%%
%% This command processes the author and affiliation and title
%% information and builds the first part of the formatted document.
\maketitle

\section{Introduction}
Krishna et al. proposed concurrent search structure templates~\cite{templates} as a method for separating the proof of correctness of a concurrent access method (optimistic concurrency, hand-over-hand locking, internal links) from the proof of correctness of the underlying data structure (linked list, hashtable, B-tree). Ideally, it should be possible to prove the correctness of $n$ (single-threaded) data structure implementations and $m$ concurrency patterns and obtain $n * m$ verified concurrent data structures. In practice, the story is more complicated: certain patterns only work for specific data structures or require the data structures to store extra information, while some internal data structure operations may not fit the template model. In this paper, we apply the template approach to concurrent data structure implementations in C, verified using the Verified Software Toolchain (VST)~\cite{plfcc}, and report on its effectiveness and the challenges we encountered. The template approach depends crucially on the idea of \emph{logically atomic specifications} introduced in TaDA~\cite{tada} and further developed in Iris~\cite{iris}; our proofs take advantage of recent work integrating Iris-style logical atomicity into VST~\cite{iris-vst-arxiv}.

\section{Background}
\subsection{Concurrent Search Structure Templates}

\subsection{Iris and VST}

\section{Search Structure Templates}
\subsection{Lock Coupling}

\subsection{Give-Up}

\subsection{Templates vs. Internal Reorganization}

\section{Verified Data Structures}
\subsection{Binary Search Tree}

\subsection{Linked List}

\subsection{Delete?}

\section{Analysis}
What could we reuse? What couldn't we?

What could we have done better?

\section{Related Work}

\section{Conclusion and Future Work}
template approach works without flow interfaces

limitations: ?

Do we get the cross-product?

lock-free implementations (and maybe weak-memory?)

\ignore{
\begin{acks}
Thanks to Roshan Sharma, Alex Oey, and Anastasiia Evdokimova for extensive work on the original binary search tree implementation and verification.
\end{acks}
}

%12 pages

%%
%% The next two lines define the bibliography style to be used, and
%% the bibliography file.
\bibliographystyle{ACM-Reference-Format}
\bibliography{../sources}


%%
%% If your work has an appendix, this is the place to put it.
%\appendix

\end{document}
\endinput
%%
%% End of file `sample-sigplan.tex'.
